\chapter{Aplicación en datos reales}

\section{ENSUSALUD 2015}
La Encuesta Nacional de Satisfacción de Usuarios del Aseguramiento Universal en Salud (ENSUSALUD) es una investigación estadística realizada por el Instituto Nacional de Estadística e Informática del Perú, cuyo objetivo es "evaluar el grado de satisfacción de los usuarios internos y externos de los servicios de salud" (INEI, 2015). La investigación ENSUSALUD es una encuesta basada en un muestreo probabilístico polietápico. La unidad primaria de muestreo (UPM) está constituida por "los establecimientos de salud del MINSA-GR. EsSalud, clínicas privadas y sanidades de las Fuerzas Armadas y Policiales" (INEI, 2015). La unidad secundaria de muestreo (USM) está constituida por los usuarios elegibles dentro del establecimiento de salud: usuarios y profesionales (de Salud y administrativos). En el caso de la UPM, el método de selección fue proporcional al tamaño, tomando en consideración el número de atenciones del establecimiento. En el caso de la USM, la selección fue aleatoria sistemática. La investigación estadística tiene el siguiente alcance:
\begin{itemize}
	\item \textbf{Cobertura geográfica:} Los 24 departamentos del Perú y 181 establecimientos de salud del MINSA, EsSalud, Sanidades y establecimientos privados.
	\item \textbf{Unidad de análisis:} La unidad muestral comprende a los siguientes:
		\begin{itemize}
			\item Usuarios de Consulta Externa.
			\item Usuarios en Boticas y Farmacias.
			\item Usuarios en Unidades de Seguros.
			\item Profesionales de la Salud.
		\end{itemize}
	\item \textbf{Niveles de inferencia:} Nacional y dirigida a cada una de las unidades de análisis.
\end{itemize}
\subsection{Base de datos}
Para propósitos de la presente investigación, se utilizó la base de datos relacionada al personal médico y de enfermería. Dicha base de datos tiene el objetivo de "conocer características del personal relacionados a formación académica, actividad laboral, satisfacción con el trabajo, estrés laboral y conocimiento referido a la SUNASA" (INEI,2015). De dicha base de datos, se utilizaron las siguientes variables:
\begin{itemize}
	\item Años de experiencia en el sector salud.
	\item Horas de trabajo semanales.
	\item Cantidad de personas que dependen económicamente del encuestado.
	\item Sexo del encuestado
	\item Límite inferior del sueldo percibido por el encuestado.
	\item Límite superior del sueldo percibido por el encuestado.
\end{itemize}

El modelo propuesto en el capítulo 3 será utilizado en la presente base de datos, tomando en consideración que el sueldo constituye una variable censurada de forma intervalar. 

\section{Resultados}

Sección en construcción.
