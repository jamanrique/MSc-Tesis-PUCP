\chapter{Introducción}

Los modelos de regresión usualmente asumen que la variable respuesta es directamente observable. No obstante, en ciertos estudios, la variable de interés no lo es. Un caso concreto de ello es el sueldo: una persona dudaría indicarle su sueldo exacto a un encuestador pues es un tema personal; tema que solo se conversa con personas de su confianza. Ante ello, el encuestador le brinda opciones de escala salarial a la persona para obtener el dato, sin comprometer la privacidad de la persona. Ante esta situación, los modelos tienen que ser adaptados a esta nueva estructura de datos y estudios.

Por otro lado, los modelos de regresión estudiados modelan la media de la variable de interés, condicionada por otro conjunto de variables. Sin embargo, el interés del investigador puede recaer en otro objetivo: más allá de la respuesta media, el investigador busca los factores subyacentes que impactan a distintos cuantiles de la variable respuesta. Los factores relacionados a una persona con un gran sueldo son distintos a una persona que no percibe mucho. Para estudios de dicho corte, los modelos de regresión cuantílica brinda la flexibilidad requerida. Dicho modelo fue propuesto inicialmente por Koenker y Basset (1978) quienes, ante la situación en dónde la estimación de mínimos cuadrados es deficiente en modelos con errores no gaussianos, proponen una regresión de cuantiles que permiten modelar libremente los cuantiles de la variable respuesta en relación a las covariables.

La presente tesis propone utilizar los temas anteriormente expuestos para implementar un modelo de regresión cuantílica aplicado a datos con censura intervalar. Para efectos de la aplicación, los datos se modelarán bajo la distribución Weibull, la cual es de amplia aplicabilidad. Dicha distribución será reparametrizada para adecuarse al modelo de regresión. Asimismo, el método de estimación será el de máxima verosimilitud,  siguiendo el marco de la inferencia clásica.

\section{Objetivos de la tesis}
El objetivo de la tesis, conforme indicado anteriormente, consiste en proponer un método de regresión cuantílica adaptado a datos con censura intervalar e implementar dicho modelo utilizando los datos de la Encuesta Nacional de Satisfacción de Usuarios en Salud. Para ello, asumimos que los datos subyacentes tienen una distribución Weibull. Los objetivos específicos son los siguientes:

\begin{itemize}
	\item Revisar literatura académica relacionada a las propuestas de modelos de regresión con datos censurados intervalarmente.
	\item Identificar una estructura apropiada de la distribución Weibull para el modelo de regresión cuantílica vía una reparametrización del modelo. Posteriormente, estudiar el comportamiento de dicha estructura.
	\item Estimar los parámetros del modelo propuesto bajo inferencia clásica.
	\item Implementar el método de estimación para el modelo propuesto en el lenguaje R y aplicarlo en datos simulados.
	\item Aplicar el modelo propuesto en datos de la Encuesta Nacional de Satisfacción de Usuarios en Salud.
\end{itemize}

\section{Organización del Trabajo}

En el capítulo 2, se presenta una estructura de la distribución Weibull, apropiada para los datos con censura intervalar. Por ello, se realiza una parametrización alternativa y se estudia los 

En el capítulo 3, se propone el modelo de regresión con datos censurados intervalarmente.

En el capítulo 4, se presenta la aplicación del modelo propuesto para determinar si existe diferencia entre los sueldos de enfermeras y enfermeros a lo largo de todos los cuantiles. Ello se realiza mediante inferencia clásica.

Finalmente, en el capítulo 5 se presentan las principales conclusiones obtenidas en la presente tesis así como los próximos pasos.