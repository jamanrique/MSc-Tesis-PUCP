\chapter{Conclusiones}
\section{Conclusiones}

Los datos con censura intervalar presentan retos en el proceso de modelamiento de datos, pues la no-observabilidad de los mismos requiere adaptar los procesos de inferencia cl�sica a esta estructura. Ante ello, la presente tesis estudi� un modelo de regresi�n cuant�lica para datos con censura intervalar, en base a los estudios realizados anteriormente por \cite{peto:p}, \cite{gentleman:lmk} y \cite{koenker:kk}. Dicho modelo de regresi�n es param�trico, asumiendo que la variable latente sigue una distribuci�n de Weibull, la cual fue reparametrizada para estudiar los efectos de las covariables en distintos cuantiles de la variable respuesta.

Para evaluar el modelo propuesto, se realiz� un estudio de simulaci�n para diversos cuantiles y distintos tama�os de muestras. Se observ� que el estimador propuesto captura apropiadamente los par�metros poblacionales, y que el sesgo y error cuadr�tico medio se redujo en la medida que aument� el n�mero de observaciones. La cobertura de los intervalos de confianza fue apropiada en todos los tama�os de muestra.

Finalmente, se aplic� el modelo de regresi�n a datos de la Encuesta Nacional de Satisfacci�n de Usuarios en Salud (ENSUSALUD) 2015. En dicha encuesta, el sueldo de los profesionales de salud (m�dicos/as y enfermeros/as) se censur� desde el proceso de recolecci�n de datos. Atendiendo al estudio realizado por \cite{salyrosas:bayes}, la presente tesis extiende el modelo de regresi�n de censura intervalar expuesto a un modelo de regresi�n cuant�lica. El presente modelo permiti� analizar los factores de las covariables en relaci�n al sueldo de dichos profesionales, por cada uno de los cuantiles de la variable respuesta.

\section{Sugerencias para investigaciones futuras}
\begin{itemize}
	\item Establecer una metodolog�a de m�xima verosimilitud que tome en cuenta el dise�o muestral a trav�s los pesos muestrales de la encuesta realizada. Asimismo, proponer m�todos para la estimaci�n de los errores est�ndar atendiendo esta estructura.
	\item Proponer un modelo de regresi�n cuant�lica con censura intervalar bajo inferencia bayesiana, tomando en consideraci�n los m�todos expuestos en la presente tesis.
\end{itemize}
