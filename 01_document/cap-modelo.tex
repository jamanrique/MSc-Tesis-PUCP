%% ------------------------------------------------------------------------- %%
\chapter{Modelo de Regresi�n Beta}
\label{cap:modregbeta}

El modelo de regresi�n para $Y_1,...,Y_n$ variables aleatorias que siguen una distribuci�n beta dada en (\ref{densidad2}) com par�metros $\mu_i$ y $\phi$ ser� obtenido cuando consideremos
\begin{equation}
g(\mu_i)=\sum_{i=1}^{k}x_{ij}\beta_j
\end{equation}

\noindent donde $\beta=(\beta_1,...,\beta_k)$ es el vector de par�metros de regresi�n, $x_{i1},...,x_{ik}$ son valores de $k$ covariables y $g(.)$ es una funci�n de enlace estrictamente mon�tona y dos veces diferenciable que va de $(0,1)$ a $\Re$. 

Desde un punto de vista cl�sico,  consideran la estimaci�n de los par�metros del modelo por el m�todo de m�xima verosimilitud. De este modo la inferencia en el modelo est� basada en propiedades asint�ticas de estos estimadores considerando una muestra suficientemente
grande.\\
Sin embargo, en muchas situaciones pr\'acticas el tama�o de muestra puede ser peque�o, o no si�ndolo suele existir alguna
informaci\'on preliminar acerca de los par�metros de inter�s. Estas situaciones son perfectamente abordados desde la perspectiva

Para la regresi\'on Beta este enfoque fue considerado por . Por este motivo, en la presente tesis se desarrollar� el an�lisis bayesiano del modelo de regresi�n beta. Entre los t�picos a ser estudiados adicionalmente consideramos: la estimaci�n del modelo utilizando MCMC, selecci�n de modelos utilizando diferente tipos de criterios y aplicaciones a conjuntos de datos reales.
