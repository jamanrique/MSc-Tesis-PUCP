\documentclass[11pt,oneside,a4paper]{book}
\usepackage[spanish]{babel}
\usepackage[latin1]{inputenc}
\usepackage[pdftex]{graphicx}           % para insertar figuras en formato pdf/png/jpg
\usepackage{color}
\usepackage{multirow}
\usepackage[table,xcdraw]{xcolor}
\usepackage{pifont}
\usepackage{amsfonts}
\usepackage{amssymb} 
\usepackage{setspace}
\usepackage[small,compact]{titlesec}
\usepackage{indentfirst} 
\usepackage[round]{natbib}
\usepackage{subfigure} 	
\usepackage[nottoc]{tocbibind}
\usepackage{setspace}
\usepackage{longtable}
\usepackage{lscape}
\usepackage{caption}
\usepackage{graphicx}
\usepackage{wrapfig}
\usepackage{lscape}
\usepackage{rotating}
\usepackage{epstopdf}
\usepackage{pdflscape}
\usepackage{amsmath}
\usepackage{listings}
\usepackage{color}
\usepackage{amsmath}
\usepackage{tcolorbox}
\usepackage{graphicx}
\usepackage{listings}
\usepackage{multirow}
\graphicspath{{../../02_figures/}}
\usepackage{longtable}
\usepackage{float}
\usepackage{adjustbox}
\pagestyle{plain}
\usepackage[colorlinks=true,urlcolor=black,citecolor=black,linkcolor=black]{hyperref}
\usepackage[a4paper,top=2.54cm, bottom=2.54cm, left=3cm, right=2.54cm]{geometry} %margenes
% ---------------------------------------------------------------------------- %
\decimalpoint
\graphicspath{{./figuras/}}
\makeindex  
\raggedbottom
\listfiles
\normalsize

\newcommand{\captionfonts}{\small}
\makeatletter  % Allow the use of @ in command names
\long\def\@makecaption#1#2{%
  \vskip\abovecaptionskip
  \sbox\@tempboxa{{\captionfonts #1: #2}}%
  \ifdim \wd\@tempboxa >\hsize
    {\captionfonts #1: #2\par}
  \else
    \hbox to\hsize{\hfil\box\@tempboxa\hfil}%
  \fi
  \vskip\belowcaptionskip}
\makeatother   % Cancel the effect of \makeatletter

\renewcommand{\topfraction}{0.85}
\renewcommand{\textfraction}{0.1}
\renewcommand{\floatpagefraction}{0.75}

% ---------------------------------------------------------------------------- %
% Cuerpo del texto
\begin{document}
\frontmatter \onehalfspacing

% ---------------------------------------------------------------------------- %

% Agradecimentos
\chapter*{Agradecimientos}
A lo largo de la escritura de esta tesis de maestr�a he recibido un inmejorable soporte y asistencia. En primer lugar, quisiera agradecer a mi asesor, el profesor Cristian Bayes, cuya supervisi�n y consejos han sido invaluables para la elaboraci�n de este documento. La retroalimentaci�n recibida mejor� la calidad de las ideas expuestas aqu�. Asimismo, agradezco nuevamente al profesor Bayes y al profesor Giancarlo Sal y Rosas. Gracias a ambos, pude iniciar mi carrera acad�mica en Estad�stica. Finalmente, agradezco a la plana docente de la maestr�a por su ense�anza y orientaci�n durante los cursos llevados.


% ---------------------------------------------------------------------------- %
% Resumen
\chapter*{Resumen}

La presente tesis propone un modelo de regresi�n cuant�lica en d�nde la variable es no negativa y posee censura intervalar, es decir que esta no es directamente observable, y la �nica informaci�n que se conoce sobre ella es que se encuentra en cierto intervalo. Para evaluar si la metodolog�a de estimaci�n captura adecuadamente los par�metros poblacionales desde el punto de vista de la inferencia cl�sica, se desarrolla un estudio de simulaci�n. Finalmente, se aplica el modelo a los datos de la Encuesta Nacional de Satisfacci�n de Salud ejecutada el a�o 2015. La estructura del modelo permite evaluar los factores relacionados al sueldo de los profesionales en salud (el cual hab�a sido censurado desde el proceso de recolecci�n de datos). El presente modelo es una extensi�n al modelo de regresi�n de censura intervalar expuesto en \cite{salyrosas:bayes}, pues eval�a los factores subyacentes a una variable respuesta a lo largo de sus cuantiles.

\noindent \textbf{Palabras clave:} censura intervalar, regresi�n con censura, inferencia cl�sica, regresi�n param�trica.

% ---------------------------------------------------------------------------- %
% Indice
\tableofcontents    % imprime el indice
% ---------------------------------------------------------------------------- %


\listoffigures               % lista de Figuras

% ---------------------------------------------------------------------------- %
\mainmatter

\onehalfspacing              % interlineado 1.5

%% ------------------------------------------------------------------------- %%
\chapter{Introducci�n}
\label{cap:introduccion}

%% ------------------------------------------------------------------------- %%
Por distintas razones, los datos recabados en una investigaci�n de �ndole estad�stica carecen de precisi�n: existen discrepancias entre el valor real del objeto de medici�n y el valor obtenido. Este proceso puede ser sist�mico: durante la administraci�n de cuestionarios a una poblaci�n objetivo, el encuestado puede omitir, reh�sar o incluso responder incorrectamente preguntas embarazosas o invasivas. Este dilema es conocido entre los encuestadores: sus encuestados, si bien est�n dispuestos a ofrecer la mejor ayuda posible, no est�n dispuestos a ofrecer informaci�n que posteriormente les pueda comprometer. Para obtener dichos datos, el encuestador usa todo su ingenio para equilibrar la privacidad del encuestado y los objetivos de su investigaci�n. En un esfuerzo de aminorar el estr�s del encuestado, el encuestador puede censurar los datos.

Este tipo de datos han sido estudiados previamente. Siguiendo las ideas de \cite{peto:p}, una variable $C$ se le denota censurada cuando su valor $c$ no se conoce y la �nica informaci�n sobre la misma es un intervalo no-cero $I$. Esta construcci�n permite definir tres tipos de datos censurados: datos censurados \textit{hacia la izquierda} (en d�nde el intervalo $I$ se define de la forma $[-\infty,L_i]$), datos censurados \textit{intervalares} (definido de la forma $[L_i, L_f]; L_i < L_f$), datos censurados \textit{hacia la derecha} (definido de la forma $[L_f, \infty]$). El presente estudio se enfoca en el segundo tipo.

Naturalmente, ello trae retos en el proceso de modelamiento de datos. Los modelos est�ndares de regresi�n presumen que la variable respuesta es directamente observable. No obstante, en situaciones como la precisada en el p�rrafo precedente dichos modelos tienen que adaptarse a la estructura de los datos. Estos modelos han sido explorados con anterioridad: \cite{gentleman:lmk} investigaron c�mo determinar la m�xima verosimilitud de los datos censurados, asegurar su consistencia e identificar m�todos algor�tmicos para su c�mputo. Utilizando los puntos de corte del dato, $L_i$ y $L_f$, era posible identificar la m�xima veros�militud a trav�s de la diferencia entre las funciones de distribuci�n acumulada en dichos puntos. Posteriormente, m�todos de regresi�n lineal atendiendo esta estructura fueron explorados por \cite{lindsey:lid} de forma param�trica. Un recuento de este tipo de m�todos se encuentra en \cite{Gomez:cal}.

Cabe resaltar que dichos m�todos de regresi�n lineal modelan la respuesta esperada de la variable respuesta condicionada por un conjunto de variables.Sin embargo, el inter�s del investigador puede recaer en otro objetivo: m�s all� de la respuesta media, el investigador busca los factores subyacentes que impactan a distintos cuantiles de la variable respuesta. Los factores relacionados a una persona con un gran sueldo son distintos a una persona que no percibe mucho. Para estudios de dicho corte, los modelos de regresi�n cuant�lica brinda la flexibilidad requerida. Dicho modelo fue propuesto inicialmente por Koenker y Basset (1978) quienes, ante la situaci�n en d�nde la estimaci�n de m�nimos cuadrados es deficiente en modelos con errores no gaussianos, proponen una regresi�n de cuantiles que permiten modelar libremente los cuantiles de la variable respuesta en relaci�n a las covariables.

La presente tesis propone utilizar los temas anteriormente expuestos para implementar un modelo de regresi�n cuant�lica aplicado a datos con censura intervalar. Para efectos de la aplicaci�n, los datos se modelar�n bajo la distribuci�n Weibull, la cual es de amplia aplicabilidad. Dicha distribuci�n ser� reparametrizada para adecuarse al modelo de regresi�n. Asimismo, el m�todo de estimaci�n ser� el de m�xima verosimilitud,  siguiendo el marco de la inferencia cl�sica. 

%% ------------------------------------------------------------------------- %%
\section{Objetivos}
\label{sec:objetivo}

El objetivo de la tesis, conforme indicado anteriormente, consiste en proponer un m�todo de regresi�n cuant�lica adaptado a datos con censura intervalar e implementar dicho modelo utilizando los datos de la Encuesta Nacional de Satisfacci�n de Usuarios en Salud. Para ello, asumimos que los datos subyacentes tienen una distribuci�n Weibull. Los objetivos espec�ficos son los siguientes:

\begin{itemize}
	\item Revisar literatura acad�mica relacionada a las propuestas de modelos de regresi�n con datos censurados intervalarmente.
	\item Identificar una estructura apropiada de la distribuci�n Weibull para el modelo de regresi�n cuant�lica v�a una reparametrizaci�n del modelo. Posteriormente, estudiar el comportamiento de dicha estructura.
	\item Estimar los par�metros del modelo propuesto bajo inferencia cl�sica.
	\item Implementar el m�todo de estimaci�n para el modelo propuesto en el lenguaje R y aplicarlo en datos simulados.
	\item Aplicar el modelo propuesto en datos de la Encuesta Nacional de Satisfacci�n de Usuarios en Salud.
\end{itemize}

%% ------------------------------------------------------------------------- %%
\section{Organizaci�n del Trabajo}
\label{sec:organizacion}

En el cap�tulo 2, se presenta una estructura de la distribuci�n Weibull, apropiada para los datos con censura intervalar. Por ello, se realiza una parametrizaci�n alternativa y se estudia los 

En el cap�tulo 3, se propone el modelo de regresi�n con datos censurados intervalarmente.

En el cap�tulo 4, se presenta la aplicaci�n del modelo propuesto para determinar si existe diferencia entre los sueldos de enfermeras y enfermeros a lo largo de todos los cuantiles. Ello se realiza mediante inferencia cl�sica.

Finalmente, en el cap�tulo 5 se presentan las principales conclusiones obtenidas en la presente tesis as� como los pr�ximos pasos.

\chapter{Distribuci�n Weibull}

El presente cap�tulo tiene como objetivo principal proponer una reparametrizaci�n de la distribuci�n Weibull para adaptarla al modelo de regresi�n cuant�lica. Para dicha reparametrizaci�n, se definir� su funci�n de densidad y funci�n acumulada, y asimismo se examinar� sus propiedades.

\section{Distribuci�n Weibull}

La distribuci�n Weibull fue presentada por \cite{weib:weib}. En dicho art�culo de investigaci�n, Weibull menciona las caracter�sticas de una funci�n de densidad suficientemente flexible para ser adaptada a diversas investigaciones, desde la rama de resistencia de materiales hasta el an�lisis de altura de hombres adultos radicados en las Islas Brit�nicas. Una variable aleatoria continua $Y$, con soporte $Y \in [0,\infty]$, sigue una distribuci�n Weibull si su funci�n de densidad es dada por la siguiente expresi�n:

\begin{equation}
f{(y)}= \frac{\alpha}{\sigma}\left(\frac{y}{\sigma}\right)^{\alpha-1} \exp{\left(-\frac{y}{\sigma}\right)^{\alpha}}
\end{equation}

\noindent en d�nde $\alpha$ corresponde al par�metro de escala, con $\alpha > 0$, y $\sigma$ corresponde al par�metro de forma, con $\sigma > 0$. La notaci�n de una variable aleatoria $Y$ que sigue esta distribuci�n se indica como $Y \sim W(\alpha,\sigma)$. La funci�n de densidad acumulada de $Y$ tiene la siguiente expresi�n:

\[F{(y)}=1-\exp^{-{(\frac{y}{\sigma})}^{\alpha}}.\]

\noindent Asimismo, su funci�n de cuantiles es dada por:

\[q_{t}=\sigma{(-\log{(1-t)})}^{\frac{1}{\alpha}}\]

\noindent para $0 < t < 1$.

La flexibilidad denotada por Weibull puede observarse a trav�s de la figura 1, en d�nde se observa que el par�metro de forma permite ..... Asimismo, el par�metro de escala permite identificar lo junta que puede ser la distribuci�n,

Para una variable $Y \sim W(\alpha, \sigma)$, la media y varianza se define de la siguiente forma:

\[E(Y)=\sigma \Gamma\left( 1+\frac{1}{\alpha} \right).\]
\[V(Y)=\sigma^{2}\left[ \Gamma\left( 1+\frac{2}{\alpha} \right)-\left( \Gamma\left( 1+\frac{1}{\alpha} \right) \right)^{2} \right].\]

\subsection{Proposici�n de una nueva estructura de la distribuci�n}

Consideramos una reparametrizaci�n del par�metro de forma $\sigma$ en t�rminos del cuantil $t, q_{t}$ en los siguientes t�rminos:

\[q_{t}=\sigma\left( -\log\left( 1-t \right) \right)^{\frac{1}{\alpha}}.\]

\noindent en d�nde $t$ ser� un valor conocido y se encuentra en el intervalo $[0,1]$. En esta nueva esctructura, $q_{t}$ y $\alpha$ tienen espacios param�tricos independientes tal que $(q_{t},\alpha) \in (0,\infty)$ x $(0,\infty)$. Una variable aleatoria que sigue esta parametrizaci�n se denota como $Y \sim W_{r}(q_{t},\alpha)$.

La funci�n de densidad de dicha variable $Y$ tiene la siguiente expresi�n:
\begin{equation}
f_{Y}(y| q_{t},\alpha)=\frac{\alpha c(t)}{q_{t}}\left( \frac{y}{q_{t}} \right)^{\alpha-1}\exp\left( -c(t)\left( \frac{y}{q_{t}} \right)^{\alpha} \right)
\end{equation}
\\
\noindent en d�nde $c(t)= \left( -\log(1-t) \right)^{\frac{1}{\alpha}}$. Los par�metros $q_{t}$ y $\alpha$ caracterizan la funci�n de densidad conforme se observa el gr�fico siguiente:

\textbf{Nota del profesor Valdivieso: Mejorar esta densidad.}

\begin{figure}
\includegraphics[width=\textwidth]{densidad1}
\caption{Funci�n de densidad de una distribuci�n Weibull bajo la reparametrizaci�n propuesta.}
\end{figure}

\noindent Se observa que en la medida que $q_{t}$ aumenta, la distribuci�n incrementa su asimetr�a hacia la derecha. Ello tambi�n sucede, aunque en menor grado, cuando $\alpha$ aumenta. No obstante, se observa que en la medida que $\alpha$ tiende a 0, incrementa la dispersi�n.

Reexpresando la funci�n acumulada en los t�rminos de la parametrizaci�n propuesta, esta tendr�a la siguiente forma:
\begin{equation} \label{eq:1}
F_{Y}\left(y| q_{t},\alpha,t \right)=1-\exp\left( -c(t)\left( \frac{y}{q_{t}} \right)^{\alpha} \right).
\end{equation}
\subsection{Estudio de la parametrizaci�n propuesta}

La esperanza y varianza de una variable aleatoria bajo la parametrizaci�n Weibull propuesta est�n dadas bajo la siguiente expresi�n:
\begin{equation}
E(Y)=\frac{q_{t}}{c(t)^{\frac{1}{\alpha}}}\Gamma\left( 1+\frac{1}{\alpha} \right)
\end{equation}

\begin{equation}
Var(Y)=\frac{q_{t}^{2}}{c(t)^{\frac{1}{\alpha}}}\left[ \Gamma\left( 1+\frac{2}{\alpha}\right)-\Gamma\left( 1+\frac{1}{\alpha} \right)^{2} \right]
\end{equation}

Bajo la parametrizaci�n propuesta, se observa que para un $\alpha$ fijo el valor esperado se comporta de forma lineal en la medida que aumente el par�metro $q_{t}$ conforme se observa en el cuadro siguiente:

\textbf{Nota del profesor Valdivieso: Mejorar esta densidad.}

\begin{figure}
	\includegraphics[width=\textwidth]{esperado}
	\caption{Valor esperado de una distribuci�n Weibull bajo la parametrizaci�n propuesta.}
\end{figure}
\noindent No obstante, para un $q_{t}$ fijo, lo mismo no se observa en la medida que aumente $\alpha$. Se observa un comportamiento no lineal y asint�tico: cuando $\alpha$ tiende a 0, el valor esperado tiende a infinito. Cuando $\alpha$ aumenta, el valor esperado se estabiliza.

En el caso de la varianza se observa que para un $\alpha$ fijo, en la medida que aumente el par�metro $q_{t}$ la varianza aumenta de forma exponencial. No obstante, y como se puede apreciar cuando $q_{t}$ est� fijo, en la medida que los valores de $\alpha$ sean peque�os, la varianza incrementa dr�sticamente. Asimismo, como se aprecia en el cuadro adjunto, la varianza tiende a 0 en la medida que $\alpha$ aumente.

\begin{figure}
	\includegraphics[width=\textwidth]{varianza}
	\caption{Varianza bajo la parametrizaci�n propuesta.}
\end{figure}

	\chapter{Modelo de regresi�n cuant�lica para datos positivos}
	El presente cap�tulo tiene como objetivo especificar el modelo de regresi�n cuant�lica para datos positivos con censura intervalar. Asimismo, detallamos la estimaci�n de sus par�metros desde la perspectiva de la inferencia cl�sica.

	\section{Datos positivos con censura intervalar}
	\label{metodo:reg}

	Siguiendo la definici�n expuesta en \cite{peto:p}, definimos a $Y$ como una variable aleatoria con una funci�n de distribuci�n acumulada $F_{Y}(y)$. Dicha variable se entiende como \textit{censurada intervalarmente} si la �nica informaci�n que tenemos sobre ella es que $Y$ yace en un intervalo $I$. Bajo este contexto, podemos definir una variable aleatoria $Z$ como una variable indicadora que precisa el $j$-�simo intervalo $[a_j,a_{j+1}[, \text{con } j = 1, \dots, k$  en el que se encuentra la variable $Y$. Por lo tanto, durante el proceso de recolecci�n de datos, observamos directamente la variable $Z$, mientras que la variable $Y$ es una variable latente. Para ilustrar este proceso, imaginemos un proceso de administraci�n de encuestas, en d�nde el encuestador consulta a la persona en qu� intervalo se encuentra su sueldo mensual. Esto requiere que la variable $Z$ sea una variable categ�rica, pues la persona solo indica una opci�n. Entonces, podemos definir dicha variable mediante la siguiente expresi�n:

	\begin{equation}
	Z = 
		\begin{cases}
			1, a_{1}< Y < a_{2} \\
			2, a_{2} \leq Y < a_{3} \\
			3, a_{3} \leq Y < a_{4} \\
			\vdots \\
			k, a_{k} \leq Y < a_{k+1} \\
		\end{cases}
	\end{equation}

	\noindent en d�nde $a_1 < a_2 < \cdots <a_{k+1}$. Estos corresponden a los l�mites del intervalo $I$, con $a_{1}=0$ y $a_{k+1}=\infty$. La funci�n de probabilidad de la variable observable $Z$ est� definida de la siguiente forma:
	\begin{equation} \label{eq:2}
		P\left( Z=j \right)=P\left(a_{j} \leq Y < a_{j+1} \right) = F_Y(a_{j+1}) - F_Y(a_{j}), j=1,\dotsc,k
	\end{equation}

	\noindent en d�nde $F_Y(\cdot)$ es la funci�n de distribuci�n acumulada de Y. La variable $Z$ que sigue la distribuci�n anteriormente mencionada est� denotada por 
\[Z \sim \text{Categ�rica}(\boldsymbol{\pi})\]

\noindent donde $\boldsymbol{\pi}=\left( \pi_{1},\dots,\pi_{k}\right)^{T}$y $\pi_{j}=P(Z=j)$.


Para efectos de la presente tesis, asumiremos que el proceso de censura de datos es independiente a la variable $Y$. \cite{calle:oller} denomina esto como un proceso no informativo, pues ello indica que el conocimiento de que una observaci�n se encuentra en el intervalo $[a_j,a_{j+1}[$ no precisa informaci�n adicional sobre la variable $Y$: solo indica que dicha variable est� contenida entre esos l�mites. Asimismo, el proceso de censura no afecta la inclusi�n de $Y$ en el intervalo correspondiente. Es decir, no existen errores u otros m�todos por los cuales $Y$ pueda pertenecer a otro intervalo que no le corresponde.

\section{Funci�n de verosimilitud para datos positivos con censura intervalar}

Bajo el contexto presentado anteriormente, y considerando las ideas plasmadas por \cite{gentleman:lmk}, el proceso de censura que deviene en la generaci�n de la variable $Z$ es independiente del proceso generador de datos de $Y$. Por lo tanto, la estimaci�n del vector de par�metros que definen la distribuci�n de $Y$, denotados por $\boldsymbol{\theta} = [q_t, \alpha]^{T}$ , no es afectado por el proceso de censura. Bajo esta suposici�n, la veros�militud de datos censurados intervalarmente (es decir, con  los datos directamente observables) es de la forma:

\begin{equation}
	L(\boldsymbol{\theta}) = \prod_{i=1}^{n} \prod_{j=1}^{k} \pi_{j}^{\mathbb{I}(Z_{i}=j)}
\end{equation}

\noindent Considerando los resultados identificados en la ecuaci�n (\ref{eq:2}), la veros�militud de la estructura observada de los datos es de la forma:

\begin{equation}
	L(\boldsymbol{\theta}) = \prod_{i=1}^{n}(F_Y(l_{i}) - F_Y(u_{i})) 
\end{equation}

\noindent en d�nde $l_i$ y $u_i$ corresponden a los l�mites inferiores y superiores del intervalo en d�nde se encuentra la $i$-�sima observaci�n.

Bajo este criterio, la veros�militud solo depende de los valores extremos del intervalo y de la funci�n de distribuci�n acumulada de la variable latente $Y$. 

\section{Modelo de regresi�n para respuestas positivas con censura intervalar}
\label{sec3.3}
Considerando la reparametrizaci�n expuesta en la secci�n 2, el modelo de regresi�n cuant�lica, basado en la distribuci�n de Weibull, est� dado por la siguiente expresi�n:

\[Y_{i} \sim W_{r}\left( q_{t_{i}},\alpha,t \right).\]
\[g\left( q_{t_{i}} \right) = \boldsymbol{x}_{i}^{T}\boldsymbol{\beta}.\]

\noindent en d�nde $\boldsymbol{\beta}=\left[ \beta_0,\beta_{1},\dots,\beta_{p} \right]^{T}$ y $\boldsymbol{x}_{i}^{T} =\left[ 1,x_{i1},x_{i2},\dots,x_{ip} \right]^{T}$, lo cual corresponde a los coeficientes y covariables respectivamente. La funci�n $g(\cdot)$ es una funci�n de enlace estrictamente mon�tona y doblemente diferenciable. En el presente modelo, se utilizar� la funci�n de enlace logar�tmica. El par�metro $\alpha$, el par�metro $q_{t_{i}}$ y $t$ est�n definidos conforme a lo visto en la secci�n \ref{sec2.2}. La estimaci�n de los par�metros $\boldsymbol{\beta}$ y $\alpha$ se realizar� mediante el m�todo de m�xima verosimilitud.

\subsection{Funci�n de verosimilitud}
\label{verofunc}

Consideramos que solo conocemos que $Y_{i}$ se encuentra en un intervalo de $K$ posibles intervalos de la forma $[a_{j},a_{j+1}[$ con $a_1 < a_2 < \dots < a_{k+1}$ y que $Z_{i}=j$ denota que $Y_{i} \in [a_{j},a_{j+1}]$. Por lo tanto, considerando los resultados de la secci�n \ref{metodo:reg}, tenemos que


\[Z_{i} \sim \text{Categ�rica}(\boldsymbol{\pi}_{i}).\]

\noindent con $\boldsymbol{\pi}_{i}=\left( \pi_{i1},\dots, \pi_{ik} \right)$ tal que

\begin{equation}
	\pi_{ij} = F_{Y}(a_{j+1}|q_{t_{i}},\alpha, t) - F_{Y}\left(a_{j}|q_{t_{i}},\alpha, t \right)
\end{equation}

\noindent d�nde $F_Y(\cdot|\cdot,\cdot)$ es la funci�n de distribuci�n acumulada de la distribuci�n Weibull reparametrizada dada en la secci�n \ref{sec2.2}. Entonces la funci�n de verosimilitud de las variables observadas $Z_{1},Z_{2},\dots,Z_{n}$ es dada por lo siguiente:

\[L(\boldsymbol{\theta})=\prod_{i=1}^{n}\prod_{j=1}^{k} \pi_{j}^{1\left( Z_{i}=j \right)}, \boldsymbol{\theta} = [\beta^{T},\alpha]^{T}.\]

\noindent Luego, considerando  [$l_{i},u_{i}$] como el intervalo d�nde $Y_{i}$ fue observado, podemos escribir la funci�n de verosimilitud como:

\[L\left( \boldsymbol{\theta}\right)=\prod_{i=1}^{n}\left( F(u_{i}|q_{t_{i}},\alpha,t) - F(l_{i}|q_{t_{i}},\alpha,t) \right) \]

As�, la funci�n de log-veros�militud es dada por:

\[l(\boldsymbol{\theta})=\sum_{i=1}^{n} \log \left( F(u_{i}|q_{t_{i}},\alpha,t) - F\left( l_{i}|q_{t_{i}},\alpha,t \right) \right)\]

\[ l(\boldsymbol{\theta})= \sum_{i=1}^{n} log\left( \left( 1-t \right)^{\left(\frac{u_i}{\exp(x_{i}^{T}\beta)}\right)^{\alpha}} - \left( 1-t \right)^{\left(\frac{l_{i}}{\exp(x_{i}^{T}\beta)}\right)^{\alpha}} \right) \]



Los estimadores de m�xima verosimilitud para los par�metros $\alpha$ y $\boldsymbol{\beta}$ se encuentran maximizando la funci�n anteriormente expuesta. Para ello, obtenemos los componentes de la gradiente la funci�n de log-veros�militud, que se presentan a continuaci�n (asumiendo que $g(\cdot)$ es la funci�n logaritmo):

\[ \frac{\partial l}{\partial \alpha}= \sum_{i=1}^{n} \frac{1}{(1-t)^{\phi_{u_i}}-(1-t)^{\phi_{l_i}}} log(1-t)(\phi_{u_i} \log(u_i\gamma_i)(1-t)^{\phi_{u_i}} - \phi_{l_i} \log(l_i\gamma_i)(1-t)^{\phi_{l_i}})\]

\[\frac{\partial l}{\partial \beta_{j}}=\sum_{i=1}^{n} \frac{\alpha x_j \log(1-t)}{(1-t)^{\phi_{u_i}} - (1-t)^{\phi_{l_i}}} (\phi_{l_i}(1-t)^{\phi_{l_i}}-\phi_{u_i}(1-t)^{\phi_{u_i}}) \]

\noindent en d�nde:
\[ \gamma_{i} = \exp(-\eta_{i})\]
\[ \eta_{i} = \boldsymbol{x}_{i}^{T}\boldsymbol{\beta}\]
\[ \phi_{u_i} = \left( \frac{u_i}{e^{x_i^{T}\beta}} \right)^{\alpha}\]
\[ \phi_{l_{i}} = \left( \frac{l_i}{e^{x_i^{T}\beta}} \right)^{\alpha}\]

Como se aprecia, no existen soluciones anal�ticas para los estimadores de m�xima veros�militud, por lo que se deben utilizar para este efecto m�todos num�ricos. Se asume que la funci�n de densidad cumple con las condiciones de regularidad expuestas en \cite{casella:berg}, por lo que los estimadores de m�xima verosimilitud identificados son consistentes (es decir, que $\boldsymbol{\hat{\theta}} \rightarrow \boldsymbol{\theta}$ cuando $n \rightarrow \infty$) y as�ntoticamente normales con distribuci�n:

\[\boldsymbol{\hat{\theta}} \sim \mathcal{N}\left(\boldsymbol{\theta},\mathcal{I}(\boldsymbol{\hat{\theta}})^{-1}\right)\]

\noindent cuando $n \rightarrow \infty$. $\mathcal{I}(\boldsymbol{\hat{\theta}})$ es la matriz de informaci�n de Fisher observada, la cual en nuestro modelo tiene la estructura:

\[
	\mathcal{I}(\boldsymbol{\hat{\theta}})=
\begin{bmatrix}

	\frac{\partial^{2} l(\boldsymbol{\theta})}{\partial \alpha \partial \alpha ^{T}} & \frac{\partial^{2} l(\boldsymbol{\theta})}{\partial \boldsymbol{\beta} \partial \alpha}\\

\frac{\partial^{2} l(\boldsymbol{\theta})}{\partial \boldsymbol{\beta} \partial \alpha} & \frac{\partial^{2} l(\boldsymbol{\theta})}{\partial \boldsymbol{\beta} \partial \boldsymbol{\beta} ^{T}}

\end{bmatrix} \rvert \boldsymbol{\theta} = \boldsymbol{\hat{\theta}}\]

\noindent Para efectos de la siguiente tesis, la evaluaci�n de esta matriz se realizar� mediante m�todos num�ricos. Los errores est�ndares de cada coeficiente se estiman a trav�s de dicha matriz de informaci�n, denotada por $\mathcal{I}(\boldsymbol{\hat{\theta}})^{-1}$. Los errores est�ndares corresponden a la ra�z cuadrada de cada elemento de la diagonal. Finalmente, en el marco de la inferencia cl�sica, los intervalos de confianza para cada par�metro est�n definidos de la forma:

\[\hat{\theta_j} \pm z_{1-\frac{\alpha}{2}} C_{jj}.\]

\noindent d�nde $C_{jj}$ es la ra�z del $j$-�simo elemento diagonal de $\mathcal{I}(\hat{\boldsymbol{\theta}})^{-1}$ y $Z_{1-\frac{\alpha}{2}}$ corresponde al valor de los l�mites $c$ tal que $P(-c < Z < c) = 1-\alpha, Z \sim N(0,1)$.

\section{Simulaci�n de datos}

En esta secci�n se presenta un estudio de simulaci�n para evaluar la metodolog�a de estimaci�n en la secci�n \ref{verofunc} permite recuperar los par�metros propuestos del modelo de censura intervalar para datos positivos. Para ello, se evaluar� el desempe�o de la simulaci�n mediante tres criterios: el sesgo relativo, el error cuadr�tico medio y el ratio de cobertura.

\subsection{Metodolog�a para la simulaci�n de datos}

La presente secci�n tiene como objetivo realizar un estudio de simulaci�n en el que se eval�e la adecuada estimaci�n del modelo propuesto. Para ello, se generar� un conjunto de datos, d�nde cada observaci�n $i$ sigue la distribuci�n $Y_i \sim W_r(q_{t_i}, \alpha,t)$. Luego, cada una de estas observaciones independientes ser�n censuradas dando como resultado la variable $Z_i$, la cual sigue lo explicado en la secci�n \ref{metodo:reg}. Asimismo, dicha base de datos contiene otras variables simuladas, las cuales actuar�n como variables independientes en un contexto de regresi�n. El objetivo principal del estudio de simulaci�n es evaluar si el m�todo de estimaci�n planteado, permite recuperar adecuadamente los par�metros de regresi�n establecidos anteriormente. Los criterios sobre los cuales se analizar� la estimaci�n del modelo son: sesgo relativo, error cuadr�tico medio y cobertura.

El proceso de simulaci�n consiste en generar $5.000$ r�plicas para cada uno de los tama�o de muestra $n \in \{100, 500, 1.000\}$. Simularemos la variable respuesta $Y_{i} \sim W_r(q_{ti},\alpha,t)$ considerando 3 covariables $X_{1i},X_{2i},X_{3i}$ que ser�n simuladas como:
\[X_{1i} \sim N(2,0.25)\]
\[X_{2i} \sim Beta(2,3)\]
\[X_{3i} \sim Gamma(2,20)\]

Conforme lo mencionado en la secci�n 3.2.1, $q_{t_{i}} =  \exp(\textbf{x}_i^T \boldsymbol{\beta})$, en d�nde $\boldsymbol{\beta} =[7, 0.3, 0.84, 2.5]^T$ y $\textbf{x}_{i}=(1,X_{1i},X_{2i},X_{3i})^{T}$. Por otro lado, el par�metro de dispersi�n tomar� el valor $\alpha = 2$. Finalmente, se realizar� la evaluaci�n por los cuantiles $t = [0.1, 0.2, \dots, 0.9]$.

Se asume que $Y_i \sim W_r(q_{ti}, \alpha,t)$ se observa con censura intervalar. En este estudio asumiremos que solo observamos una variable $Z$ que particiona la variable $Y_i$ en intervalos de igual amplitud, con la excepci�n del �ltimo intervalo, el cual tiene la estructura $[a_{j}, \infty)$. Una vez generada dicha variable, se realiza el modelamiento de la variable con censura intervalar sobre las variables independientes creadas previamente. El objetivo final es, a trav�s del m�todo de m�xima veros�militud, estimar los coeficientes $\boldsymbol{\beta}$ y $\alpha$ definidos previamente.

\subsection{Implementaci�n del modelo}

La implementaci�n del modelo se realiz� a trav�s del lenguaje de programaci�n R, tomando en consideraci�n las definiciones presentadas en el cap�tulo 3 de la presente tesis. Asimismo, se utiliz� paquetes de optimizaci�n um�rica como \textit{nloptr} para identificar los estimados de m�xima veros�militud. El pseudoc�digo de la implementaci�n se encuentra en el Ap�ndice.

Una vez generadas las simulaciones, se evalu� para cada escenario (cuantil y tama�o de muestra) los siguientes indicadores:
\[ \hat{\text{Sesgo relativo:}} \frac{1}{M}\sum_{j=1}^{M}\frac{(\hat{\theta_j} - \theta)}{\theta}\]
\[ \hat{\text{ECM:} \frac{1}{M}} \sum_{j=1}^M (\hat{\theta_j} - \theta)^2 \]
\[ \hat{\text{Cobertura:} \frac{1}{M}} \sum_{j=1}^M I(\boldsymbol{\theta} \in IC_{j})\]

\noindent d�nde $\theta$ es el verdadero valor del par�metro, $\hat{\theta}_{j}$ la estimaci�n obtenida en la $j$-�sima r�plica, M es el n�mero de r�plicas, $IC_j$ es el intervalo de confianza al 95\% obtenido en la $j$-�sima r�plica, y $I(\boldsymbol{\theta} \in IC_j)$ es la funci�n indicadora que identifica si el verdadero valor del par�metro se encuentra en el intervalo de confianza obtenido en la $j$-�sima r�plica.

\subsection{Resultados}

En las figuras \ref{fig:ses}, \ref{fig:ecm}, y \ref{fig:cob} se muestran la evaluaci�n del rendimiento del modelo de regresi�n cuant�lica con censura intervalar, de acuerdo a los criterios expuestos anteriormente. Observamos lo siguiente:

\begin{itemize}


	\item En relaci�n al sesgo relativo, se observa que este disminuye a lo largo de todos los par�metros en la medida que aumenta el tama�o de la muestra. Cabe resaltar que para tama�os de muestra peque�os, el par�metro $\alpha$ tiende a sobre-estimarse, no obstante esto disminuye considerablemente en la medida que el tama�o de muestra aumente.

	\item En relaci�n a la cobertura, se observa que, para todos los tama�os de muestra, los par�metros establecidos en la secci�n precedente se encuentran aproximadamente el 95\% de las veces dentro del intervalo de confianza generado.

	\item En relaci�n al error cuadr�tico medio, se observa que, para un tama�o de muestra peque�o, el error es considerable para todos los par�metros. No obstante, esto disminuye dr�sticamente en la medida que el tama�o de muestra aumenta.

\end{itemize}

\begin{figure}
\centering
	\includegraphics[width=\textwidth]{Sesgo}
	\caption{Estudio de Simulaci�n: An�lisis del sesgo}
	\label{fig:ses}
\end{figure}

\begin{figure}
\centering
\includegraphics[width=\textwidth]{ECM}
	\caption{Estudio de Simulaci�n: An�lisis del error cuadr�tico medio}
	\label{fig:ecm}
\end{figure}

\begin{figure}
\centering
	\includegraphics[width=\textwidth]{Cobertura}
	\caption{Estudio de Simulaci�n: An�lisis de la Cobertura}
	\label{fig:cob}
\end{figure}

\chapter{Aplicaci�n en datos reales}

El presente cap�tulo aplica el modelo propuesto en el cap�tulo 3 a una encuesta de satisfacci�n de profesionales peruanos del sector salud. Se busca estimar los efectos de cada uno de los atributos en relaci�n al sueldo reportado de dichos profesionales, con especial �nfasis en el sexo de los mismos. Este �nfasis est� sustentado en estudios previos de instituciones del Gobierno del Per� y acad�micos. Al respecto, recientes investigaciones del INEI \footnote{Instituto Nacional de Estad�stica e Inform�tica del Per�.} identifican una brecha promedio de 29\% entre los sueldos de mujeres y hombres, siendo estos �ltimos quienes ganan m�s. Bajo dicha premisa, \cite{salyrosas:bayes} tambi�n efectuaron el an�lisis a encuestas de profesionales de salud, identificando tambi�n dicha brecha. La presente aplicaci�n tiene el objetivo de brindar una figura m�s completa de los efectos de las covariables por cada uno de los cuantiles del sueldo reportado, ampliando el an�lisis expuesto en \cite{salyrosas:bayes}.

\section{Sobre los datos utilizados}
Durante los a�os 2013 al 2015, instituciones gubernamentales del Per�, el INEI y la SUNASA \footnote{Superintendencia Nacional de Aseguramiento en Salud.} acordaron implementar y ejecutar encuestas nacionales de satisfacci�n en el sector salud. Ello comprende todas las ramas involucradas de dicho sector: los usuarios del servicio (de consulta externa, de boticas y farmacias, y de unidades de seguros) as� como profesionales de la salud (m�dicos y enfermeros), por cada una de los conjuntos de establecimientos que existen. Estos comprenden establecimientos del Ministerio de Salud, Seguro Social de Salud, cl�nicas privadas, y establecimientos de Sanidad de las Fuerzas Armadas y Policiales. El objetivo final de la encuesta fue <<evaluar el grado de satisfacci�n de los usuarios internos y externos de los servicios de salud>> \cite{inei:ensusalud}. 

Dicha encuesta form� parte de las investigaciones estad�sticas realizadas por el INEI, cuyo dise�o muestral fue probabil�stico y poliet�pico. La primera unidad de muestreo constituy� el establecimiento de salud por cada uno de los conjuntos anteriormente expuestos, y la segunda unidad de muestreo constiti� en los usuarios elegibles y profesionales de la salud. La encuesta tuvo alcance nacional, por los 24 departamentos del Per�. El nivel de inferencia indicado por el INEI es nacional y dirigida a cada una de las unidades de an�lisis.

Para pr�positos de la aplicaci�n, se utiliz� la encuesta ejecutada a profesionales de la salud. En esta, los datos obtenidos est�n relacionados con la formaci�n acad�mica, actividad laboral, satisfacci�n en el trabajo, estr�s laboral y conocimiento de la SUNASA \cite{inei:ensusalud}. 

De dicha encuesta, se utilizaron las siguientes variables:

% Please add the following required packages to your document preamble:
% \usepackage{multirow}
% \usepackage{graphicx}
% \usepackage[table,xcdraw]{xcolor}
% If you use beamer only pass "xcolor=table" option, i.e. \documentclass[xcolor=table]{beamer}
\begin{table}[]
\centering
\resizebox{\textwidth}{!}{%
\begin{tabular}{|c|l|l|}
\hline
\rowcolor[HTML]{C0C0C0} 
\textbf{Grupo de Variable} &
  \multicolumn{1}{c|}{\cellcolor[HTML]{C0C0C0}\textbf{Descripci�n de la variable}} &
  \multicolumn{1}{c|}{\cellcolor[HTML]{C0C0C0}\textbf{C�digo}} \\ \hline
Establecimiento de Salud &
  Tipo de instituci�n del establecimiento &
  INSTITUCION \\ \hline
\begin{tabular}[c]{@{}c@{}}Formaci�n del Profesional \\ de Salud\end{tabular} &
  �El profesional cuenta con especialidad? &
  C2P13 \\ \hline
 &
  \begin{tabular}[c]{@{}l@{}}A�os de experiencia en el sector\\ salud\end{tabular} &
  C2P21 \\ \cline{2-3} 
 &
  \begin{tabular}[c]{@{}l@{}}�El profesional realiza labor asistencial\\ en otra instituci�n?\end{tabular} &
  C2P24 \\ \cline{2-3} 
 &
  \begin{tabular}[c]{@{}l@{}}�El profesional realiza labor docente\\ remunerada?\end{tabular} &
  C2P26 \\ \cline{2-3} 
\multirow{-4}{*}{\begin{tabular}[c]{@{}c@{}}Actividad Laboral del Profesional\\ de salud\end{tabular}} &
  \begin{tabular}[c]{@{}l@{}}Cantidad de horas laboradas semanalmente\\ por el profesional de salud\end{tabular} &
  C2P27 \\ \hline
 &
  Sexo del profesional de salud &
  C2P4 \\ \cline{2-3} 
\multirow{-2}{*}{Atributos del Profesional de Salud} &
  Rango Salarial &
  {[}li, lf{]} \\ \hline
\end{tabular}%
}
\caption{Descripci�n de las variables dentro de la base de datos}
\label{tab:tablitadescripcion}
\end{table}

\subsection{An�lisis descriptivo de los datos}

En la figura \ref{fig:sueldos} se puede observar que existe una mayor proporci�n de profesionales m�dicos varones en los intervalos salariales superiores que profesionales mujeres por cada una de las instituciones de la encuesta. Cabe resaltar que se observa una mayor proporci�n de encuestados en los establecimientos del Ministerio de Salud y ESSALUD, no obstante la proporci�n de varones en la banda salarial m�s alta se mantiene a lo largo de todos los establecimientos. Asimismo, en la figura \ref{fig:anos} se puede observar que para las bandas salariales m�s bajas, las mujeres tienen m�s a�os de experiencia que los hombres (observando la mediana); no obstante se mantienen en la misma banda salarial. Lo anteriormente precisado sugiere que las mujeres tienen una menor probabilidad de pertenecer a las bandas salariales superiores. No obstante, esto se verificar� a trav�s de la aplicaci�n del modelo de censura intervalar.

\begin{figure}[H]
	\includegraphics[width=\textwidth]{figuras/distribsueldos.jpeg}
	\caption{Distribuci�n de sueldos por sexo e instituci�n}
\label{fig:sueldos}
\end{figure}

\begin{figure}[H]
	\includegraphics[width=\textwidth]{figuras/distribanos.jpeg}
	\caption{A�os de experiencia por sexo y banda salarial}
	\label{fig:anos}
\end{figure}

\section{Resultados}

Tomando como variable respuesta la banda salarial, se ajust� el modelo de regresi�n cuant�lica para datos intervalares a los datos. Dado que existen variables categ�ricas en la base de datos, las categor�as de referencia para el presente estudio es un m�dico hombre, con labor docente remunerada,tiene estudios de especializaci�n, que trabaja en un establecimiento del MINSA, y adicionalmente realizar�a labor asistencial en otra instituci�n. En ese sentido, el intervalo puede ser interpretado como una estimaci�n del sueldo de dicha categor�a de referencia a lo largo de los cuantiles de la variable respuesta. Para el ajuste del modelo, se consider� una funci�n de enlace logar�tmica para los par�metros $\boldsymbol{\beta}$ y $\alpha$. 

La figura \ref{med:efectos} presenta un resumen de los resultados de la aplicaci�n del modelo y el ap�ndice muestra las estimaciones por cada cuantil. Cada uno de los gr�ficos presentados tiene como eje horizontal el nivel del cuantil $\tau$, y como eje vertical el efecto marginal de la covariable. En dicho gr�fico, se tiene tanto la estimaci�n por el modelo de regresi�n cuant�lica para datos con censura intervalar y un modelo de regresi�n censurada. En azul, se tienen los efectos estimados de cada covariable por el modelo propuesto en el cap�tulo 3, as� como sus intervalos de confianza al 95\%. En negro y gris, se tiene la estimaci�n mediante un modelo de regresi�n de Weibull que estima la media.

En relaci�n al efecto medio, se observa que un profesional de salud mujer tendr�a un decremento salarial en relaci�n a la categor�a de referencia. No obstante, el an�lisis se enriquece cuando tomamos en consideraci�n el modelo de regresi�n cuant�lica para datos intervalares, pues se observa que para los cuantiles inferiores existe un considerable efecto negativo para dicha variable. Esta disparidad entre el efecto medio y los efectos por cada cuantil se pueden apreciar adicionalmente en la cantidad de horas trabajadas (C2P27), pues se observa que el efecto positivo de las horas trabajadas es mayor en la medida que se analiza los cuantiles superiores. Por otro lado, el efecto negativo de no contar con una especializaci�n (C2P13) es cada vez menor en la medida que se eval�an los cuantiles superiores.

\begin{figure}[H]
\includegraphics[width=\textwidth]{figuras/regresion_2.jpeg}
\caption{Efectos de las covariables (eje vertical) sobre los cuantiles (eje horizontal) del sueldo de los profesionales de la salud.}
\label{med:efectos}

\end{figure}

\chapter{Conclusiones}
\section{Conclusiones}

Los datos con censura intervalar presentan retos en el proceso de modelamiento de datos, pues la no-observabilidad de los mismos requiere adaptar los procesos de inferencia cl�sica a esta estructura. Ante ello, la presente tesis estudi� un modelo de regresi�n cuant�lica para datos con censura intervalar, atendiendo los estudios realizados anteriormente por \cite{peto:p}, \cite{gentleman:lmk} y \cite{koenker:kk}. Dicho modelo de regresi�n es param�trico, asumiendo que la variable latente sigue una distribuci�n Weibull, la cual fue reparametrizada para estudiar los efectos de las covariables en distintos cuantiles de la variable respuesta.

Para evaluar el modelo propuesto, se realiz� un estudio de simulaci�n para todos los cuantiles y distintos niveles de muestras. Se observ� que el modelo propuesto captura apropiadamente los par�metros poblacionales, y que el sesgo y error cuadr�tico medio se reduci� en la medida que aument� el n�mero de observaciones. La cobertura de los par�metros fue apropiada en todos los niveles de muestra.

Finalmente, se aplic� el modelo de regresi�n a datos de la Encuesta Nacional de Satisfacci�n de Usuarios en Salud (ENSUSALUD) 2015. En dicha encuesta, el sueldo de los profesionales de salud (m�dicos/as y enfermeros/as) se censur� desde el proceso de recolecci�n de datos. Atendiendo al estudio realizado por \cite{salyrosas:bayes}, la presente tesis extiende el modelo de regresi�n de censura intervalar expuesto a un modelo de regresi�n cuant�lica. El presente modelo permiti� analizar los factores de las covariables en relaci�n al sueldo de dichos profesionales, por cada uno de los cuantiles de la variable respuesta.

\section{Sugerencias para investigaciones futuras}
\begin{itemize}
	\item Establecer un m�todo de verosimilitud que tome en cuenta el dise�o muestral de la encuesta realizada. Asimismo, proponer m�todos de estimaci�n de varianza atendiendo esta estructura.
	\item Proponer un modelo de regresi�n cuant�lica con censura intervalar bajo inferencia bayesiana, tomando en consideraci�n los m�todos de verosimilitud expuestos en la presente tesis.
\end{itemize}

\chapter{Ap�ndice}

\section{Pseudoc�digo de la simulaci�n}
\label{seudo}

\begin{lstlisting}
Simulamos valores de las siguientes distribuciones:

Definimos los siguientes valores:
N = [100, 500, 1000]
B = [7, 0.3, 0.84, 2.5] 
Sigma = 2
t=[0.1, 0.2, 0.3, 0.4, 0.5, 0.6, 0.7, 0.8, 0.9]
M = 5000

Para cada cuantil en t:
	Para cada n en N:
		Para cada replica en M:
		1 Simular n valores de las siguientes distribuciones:
			X1 ~ Beta(2,3) 
			X2 ~ Normal(2,0.5)
			X3 ~ Gamma(2,25)
		2 Generar la funci�n de enlace:
			Qt = exp(B[1] + B[2]*X1 + B[3]*X2 + B[4]*X3)
		3 Para cada i en n:
			Simular 1 valor de la siguiente distribucion:
			Y[i] ~ W_r(Qt[i], Sigma, cuantil)
		4 Censurar la variable Y de forma intervalar tal que
			Z ~ Categorica
		5 Obtener los limites inferiores y superiores de
			cada categoria de Z
		6 Crear la base de datos simulada
			df <- [L_inf, L_sup, X1, X2, X3]
		7 Ejecutar la regresion de censura intervalar
		8 Guardar los resultados
\end{lstlisting}

\section{Aplicaci�n en R}

\begin{lstlisting}[basicstyle=\tiny]
library(gamlss)
library(gamlss.cens)
library(haven)
library(tidyverse)
library(BB)
library(matrixStats)
library(gridExtra)
library(numDeriv)
library(pracma)
library(ucminf)
library(nloptr)
library(ggthemes)

gen.cens(family="WEI3",type="interval")

Ct = function(alpha, tau){
  (-log(1-tau))^(1/alpha)
}
F_Wr = function(Y,Qt,alpha,tau){
  B = Qt/Ct(alpha,tau)
  pweibull(Y,shape=alpha,scale=B)
}
Qt_b = function(betas, df){
  Qt_c = exp(as.matrix(df[['matriz.diseno']]) %*% betas)
  return(Qt_c)
}
Qt_a = function(betas, df){
  Qt_c = exp(as.matrix(df) %*% betas)
  return(Qt_c)
}
Rand_Wr = function(n,Qt,alpha,tau){
  B = Qt/Ct(alpha,tau)
  rweibull(n,shape = alpha,scale = B)
}
Simulation = function(n){
  X1 = rnorm(n,2,0.25)
  X2 = rbeta(n,2,3)
  X3 = rgamma(n,2,20)
  df = data.frame(X1 = X1, X2 = X2, X3 = X3)
  return(df)
}
DF_Simulation = function(df,betas,alpha,tau){
  M = dim(df)[1]
  design_matrix = model.matrix(~ . ,df)
  Qt_i = Qt_a(betas,design_matrix)
  Y = c()
  for (j in 1:M) {
    Y=rbind(Y,Rand_Wr(1,Qt_i[j],alpha,tau))
  }
  min_Y = floor(min(Y))
  if (min_Y==0) {
    min_Y <- 0.01
  }
  Q8_Y = round(quantile(Y,0.8),2)
  interval = round((Q8_Y-min_Y) / 6,2)
  seq_interv = c(seq(min_Y,Q8_Y,interval),Inf)
  Ls = c()
  for (u in 1:length(Y)) {
    for (n in 1:length(seq_interv)) {
      if (Y[u] <seq_interv[n]) {
        Ls[u] = seq_interv[n]
        break
      }}}
  Li = c()
  for (p in 1:length(Y)) {
    for (w in 1:length(seq_interv)) {
      if (Y[p] > rev(seq_interv)[w]) {
        Li[p] = rev(seq_interv)[w]
        break
      }}}
  F_df = cbind(data.frame(Li = round(Li,2), Ls = round(Ls,2)),df)
  return(F_df)
}
data_mgmt = function(data,li,lf){
  df_inf = data[,li]
  df_sup = data[,lf]
  covar = as.data.frame(data[,-c(li,lf)])
  covar = model.matrix(~. , covar)
  lista = list(lim.inferior = df_inf, lim.superior = df_sup, matriz.diseno = covar)
  return(lista)
}
likelihood = function(param,betas,df,tau){
  n = length(param)-1
  for(i in 1:n){betas[i]=param[i]}
  alpha = param[length(param)]
  Qt_i = Qt_b(betas,df)
  -sum(log(F_Wr(df[['lim.superior']],Qt_i,alpha,tau)
       -F_Wr(df[['lim.inferior']],Qt_i,alpha,tau)))
}
reg_Wr = function(data,li,lf,tau,param){
  df = data_mgmt(data,li,lf)
  Bs = as.matrix(rep(0,ncol(df[['matriz.diseno']])))
  fit_mv = nloptr(x0 = param,eval_f = likelihood, betas=Bs, df=df, tau=tau,
  opts = list("algorithm"=c("NLOPT_LN_SBPLX"),maxeval = 400)
  ,lb = c(-Inf,-Inf,-Inf,-Inf,-Inf,-Inf,-Inf,-Inf,-Inf,-Inf,-Inf)
  ,ub=c(Inf,Inf,Inf,Inf,Inf,Inf,Inf,Inf,Inf,Inf,Inf))
  print(fit_mv$message)
  fit_mv$hessian = pracma::hessian(x0 = fit_mv$solution,f = likelihood,betas=Bs,df=df,tau=tau)
  return(fit_mv)
}

lancet <- function(int){
  temp <- tempfile()
  download.file("http://iinei.inei.gob.pe/iinei/srienaho/descarga/SPSS/447-Modulo552.zip", temp)
  salud <- read_sav(unz(temp, "447-Modulo552/04_C2_CAPITULOS.sav"))
  unlink(temp)
  
  salud <- salud[salud$C2P28 != 7,]
  
  salud$C2P26
  
  ### Pre-procesamiento ###
  variables <- c("INSTITUCION","C2P4","C2P13","C2P21","C2P24","C2P26","C2P28","C2P1","C2P27")
  
  newsalud <- salud[variables]
  
  newsalud$li <- NULL
  newsalud$ls <- NULL
  newsalud$li <- ifelse(newsalud$C2P28 == 1, 850,
                        ifelse(newsalud$C2P28 == 2, 1000,
                               ifelse(newsalud$C2P28 == 3, 2001,
                                      ifelse(newsalud$C2P28 == 4, 3001,
                                             ifelse(newsalud$C2P28 == 5, 4001, 5001)))))
  newsalud$lf <- ifelse(newsalud$C2P28 == 1, 999,
                        ifelse(newsalud$C2P28 == 2, 2000,
                               ifelse(newsalud$C2P28 == 3, 3000,
                                      ifelse(newsalud$C2P28 == 4, 4000,
                                             ifelse(newsalud$C2P28 == 5, 5000, Inf)))))
  
  newsalud <- newsalud %>% select(-C2P28)
  
  newsalud$INSTITUCION <- factor(newsalud$INSTITUCION,labels=names(attr(newsalud$INSTITUCION,"labels")))
  newsalud$C2P4 <- factor(newsalud$C2P4,labels = names(attr(newsalud$C2P4,"labels")))
  newsalud$C2P13 <- factor(newsalud$C2P13, labels=names(attr(newsalud$C2P13,"labels")))
  newsalud$C2P21 <- as.integer(newsalud$C2P21)
  newsalud$C2P24 <- factor(newsalud$C2P24, labels = names(attr(newsalud$C2P24,"labels")))
  newsalud$C2P26 <- factor(newsalud$C2P26, labels = names(attr(newsalud$C2P26,"labels")))
  newsalud$C2P1 <- factor(newsalud$C2P1, labels = names(attr(newsalud$C2P1,"labels")))
  newsalud$C2P27 <- as.integer(newsalud$C2P27)
  
  newsalud_enf <- newsalud[newsalud$C2P1 =="Enfermero/a" ,]
  newsalud_enf <- subset(newsalud_enf, select = -C2P1)
  newsalud_enf <- as.data.frame(newsalud_enf)
  
  newsalud_med <- newsalud[newsalud$C2P1 == "M�dico",]
  newsalud_med <- subset(newsalud_med, select = -C2P1)
  newsalud_med <- as.data.frame(newsalud_med)
  if (int == 1) {
    return(newsalud_enf)  
  }
  if (int ==2) {
    return(newsalud_med)
  }
  if (int ==3) {
    newsalud_total <- as.data.frame(subset(newsalud,select = -C2P1))
    return(newsalud_total)
  }
}

#### Simulaci�n ####

L = 5000
n = c(100,500,1000)
betas_sim = c(7,0.3,0.84,2.5)
alpha_sim = 2
tau_sim = seq(0.1,0.9,0.1)

sim_list= list(n100 = list(), n500 = list(), n1000 = list())

for (j in 1:length(n)) {
  for (k in 1:length(tau_sim)) {
    for (p in 1:L) {
      sim = Simulation(n[j])
      df_sim = DF_Simulation(sim,betas_sim,alpha_sim,tau_sim[k])
      m0 = gamlss(Surv(Li,Ls,type="interval2")~.,family = WEI3ic,data = na.omit(df_sim))
      init = as.vector(c(coef(m0),m0$sigma.coefficients))
      sim_list[[j]] = append(sim_list[[j]],list(reg_Wr(data = df_sim,
                       li = 1,lf = 2,tau = tau_sim[k],param = init)))
    }
  }
}

se_calc <- function(lista) {
  ic_contain <- matrix(nrow=0,ncol=5)
  for (k in 1:length(lista)) {
    little_t <- lista[[k]]
    ic <- rbind( little_t$par - qnorm(0.975) * sqrt(diag(solve(little_t$hessian))),
                 little_t$par + qnorm(0.975) * sqrt(diag(solve(little_t$hessian))))
    ic_contain <- rbind(ic_contain,cbind(between(betas_sim[1],left = ic[1,1],right = ic[2,1]),
                        between(betas_sim[2],left = ic[1,2],right = ic[2,2]),
                        between(betas_sim[3],left = ic[1,3],right = ic[2,3]),
                        between(betas_sim[4],left = ic[1,4],right = ic[2,4]),
                        between(alpha_sim,left = ic[1,5],right = ic[2,5])))
  }
  final_df <- as.matrix(t(colSums(ic_contain)/length(lista)))
  return(final_df)
}

final_database <- matrix(nrow=0,ncol=5)
seq_t <- seq(1,8,1);seq_t

for (j in 1:length(n)) {
  df <- sim_list[[j]]
  actual_t <- df[1:5000*seq_t[1]]
  vf_df <- se_calc(actual_t)
  rownames(vf_df) <- paste("n_",n[j],sep="")
  final_database <- rbind(final_database,vf_df)
  
  for (l in 1:length(seq_t)) {
   actual_t <-  df[(5000*seq_t[l]+1):(5000*(seq_t[l]+1))]
   vf_df <- se_calc(actual_t)
   rownames(vf_df) <- paste("n_",n[j],sep="")
   final_database <- rbind(final_database,vf_df)
  }
}

final_database <- as.data.frame(final_database)

final_database_vf <- cbind(
  Cuantil = rep(c('0.1','0.2','0.3','0.4','0.5','0.6','0.7','0.8','0.9'),3),
  B1 = c(rep(100,9),rep(500,9),rep(1000,9)), 
  final_database)

v1 <- ggplot(data = final_database_vf) +
  aes(x = "", y = V1) +
  geom_boxplot(shape = "circle", fill = "#B22222") +
  facet_wrap(vars(B1)) +
  scale_color_brewer(palette = "YlOrRd") + 
  labs(x="Tama�o de Muestra", y="Cobertura",title = "Cobertura para \U003B2_0") +
  scale_y_continuous(limits=c(0.93,0.96)) + 
  geom_hline(yintercept=0.95,linetype="dashed",color="red")+
  theme_minimal() +
  theme(legend.position="bottom")

v2 <- ggplot(data = final_database_vf) +
  aes(x = "", y = V2) +
  geom_boxplot(shape = "circle", fill = "#B22222") +
  facet_wrap(vars(B1)) +
  scale_color_brewer(palette = "YlOrRd") + 
  labs(x="Tama�o de Muestra", y="Cobertura",title = "Cobertura para \U003B2_1") +
  scale_y_continuous(limits=c(0.93,0.96)) + 
  geom_hline(yintercept=0.95,linetype="dashed",color="red")+
  theme_minimal() +
  theme(legend.position="bottom")

v3 <- ggplot(data = final_database_vf) +
  aes(x = "", y = V3) +
  geom_boxplot(shape = "circle", fill = "#B22222") +
  facet_wrap(vars(B1)) +
  scale_color_brewer(palette = "YlOrRd") + 
  labs(x="Tama�o de Muestra", y="Cobertura",title = "Cobertura para \U003B2_2") +
  scale_y_continuous(limits=c(0.93,0.96)) + 
  geom_hline(yintercept=0.95,linetype="dashed",color="red")+
  theme_minimal() +
  theme(legend.position="bottom")

v4 <- ggplot(data = final_database_vf) +
  aes(x = "", y = V4) +
  geom_boxplot(shape = "circle", fill = "#B22222") +
  facet_wrap(vars(B1)) +
  scale_color_brewer(palette = "YlOrRd") + 
  labs(x="Tama�o de Muestra", y="Cobertura",title = "Cobertura para \U003B2_3") +
  scale_y_continuous(limits=c(0.93,0.96)) + 
  geom_hline(yintercept=0.95,linetype="dashed",color="red")+
  theme_minimal() +
  theme(legend.position="bottom")

v5 <- ggplot(data = final_database_vf) +
  aes(x = "", y = V5) +
  geom_boxplot(shape = "circle", fill = "#B22222") +
  facet_wrap(vars(B1)) +
  scale_color_brewer(palette = "YlOrRd") + 
  labs(x="Tama�o de Muestra", y="Cobertura",title = "Cobertura para \U003B1") +
  scale_y_continuous(limits=c(0.93,0.96)) + 
  geom_hline(yintercept=0.95,linetype="dashed",color="red")+
  theme_minimal() +
  theme(legend.position="bottom")

ggpubr::ggarrange(v1,v2,v3,v4,v5)

ecm_calc <- function(lista) {
  ic_contain <- matrix(nrow=0,ncol=5)
  for (k in 1:length(lista)) {
    little_t <- lista[[k]]
    ecm <- (little_t$par - c(betas_sim,alpha_sim))**2
    ic_contain <- rbind(ic_contain,ecm)
  }
  final_df <- as.matrix(t(colSums(ic_contain)/length(lista)))
  return(final_df)
}

  final_database <- matrix(nrow=0,ncol=5)

seq_t <- seq(1,8,1);seq_t

for (j in 1:length(n)) {
  df <- sim_list[[j]]
  actual_t <- df[1:5000*seq_t[1]]
  vf_df <- ecm_calc(actual_t)
  rownames(vf_df) <- paste("n_",n[j],sep="")
  final_database <- rbind(final_database,vf_df)
  
  for (l in 1:length(seq_t)) {
    actual_t <-  df[(5000*seq_t[l]+1):(5000*(seq_t[l]+1))]
    vf_df <- ecm_calc(actual_t)
    rownames(vf_df) <- paste("n_",n[j],sep="")
    final_database <- rbind(final_database,vf_df)
  }
}
round(final_database,4)

final_database <- as.data.frame(final_database)

final_database_vf <- cbind(
  Cuantil = rep(c('0.1','0.2','0.3','0.4','0.5','0.6','0.7','0.8','0.9'),3),
  B1 = c(rep(100,9),rep(500,9),rep(1000,9)), 
  final_database)

v1 <- ggplot(data = final_database_vf) +
  geom_line(aes(x=B1,y=V1,color=Cuantil,group=Cuantil),size=1.2) +
  scale_color_brewer(palette = "YlOrRd") + 
  labs(x="Tama�o de Muestra", y="ECM",title = "ECM para \U003B2_0") +
  scale_x_continuous(breaks=c(100,500,1000)) +
  theme_minimal() +
  theme(legend.position="bottom")


v2 <- ggplot(data = final_database_vf) +
  geom_line(aes(x=B1,y=V2,color=Cuantil,group=Cuantil),size=1.2) +
  scale_color_brewer(palette = "YlOrRd") + 
  labs(x="Tama�o de Muestra", y="ECM",title = "ECM para \U003B2_1") +
  scale_x_continuous(breaks=c(100,500,1000)) + 
  theme_minimal() +
  theme(legend.position="bottom")

v3 <- ggplot(data = final_database_vf) +
  geom_line(aes(x=B1,y=V3,color=Cuantil,group=Cuantil),size=1.2) +
  scale_color_brewer(palette = "YlOrRd") + 
  labs(x="Tama�o de Muestra", y="ECM",title = "ECM para \U003B2_2") +
  scale_x_continuous(breaks=c(100,500,1000)) + 
  theme_minimal() +
  theme(legend.position="bottom")

v4 <- ggplot(data = final_database_vf) +
  geom_line(aes(x=B1,y=V4,color=Cuantil,group=Cuantil),size=1.2) +
  scale_color_brewer(palette = "YlOrRd") + 
  labs(x="Tama�o de Muestra", y="ECM",title = "ECM para \U003B2_3") +
  scale_x_continuous(breaks=c(100,500,1000)) + 
  theme_minimal() +
  theme(legend.position="bottom")

v5 <- ggplot(data = final_database_vf) +
  geom_line(aes(x=B1,y=V5,color=Cuantil,group=Cuantil),size=1.2) +
  scale_color_brewer(palette = "YlOrRd") + 
  labs(x="Tama�o de Muestra", y="ECM",title = "ECM para \U003B1") +
  scale_x_continuous(breaks=c(100,500,1000)) + 
  theme_minimal() +
  theme(legend.position="bottom")

ggpubr::ggarrange(v1,v2,v3,v4,v5)

ses_calc <- function(lista) {
  ic_contain <- matrix(nrow=0,ncol=5)
  for (k in 1:length(lista)) {
    little_t <- lista[[k]]
    ecm <- (little_t$par - c(betas_sim,alpha_sim))
    ic_contain <- rbind(ic_contain,ecm)
  }
  final_df <- as.matrix(t(colSums(ic_contain)/length(lista)))
  return(final_df)
}

final_database <- matrix(nrow=0,ncol=5)

seq_t <- seq(1,8,1);seq_t

for (j in 1:length(n)) {
  df <- sim_list[[j]]
  actual_t <- df[1:5000*seq_t[1]]
  vf_df <- ses_calc(actual_t)
  rownames(vf_df) <- paste("n_",n[j],sep="")
  final_database <- rbind(final_database,vf_df)
  
  for (l in 1:length(seq_t)) {
    actual_t <-  df[(5000*seq_t[l]+1):(5000*(seq_t[l]+1))]
    vf_df <- ses_calc(actual_t)
    rownames(vf_df) <- paste("n_",n[j],sep="")
    final_database <- rbind(final_database,vf_df)
  }
}

final_database <- as.data.frame(final_database)

final_database_vf <- cbind(
  Cuantil = rep(c('0.1','0.2','0.3','0.4','0.5','0.6','0.7','0.8','0.9'),3),
  B1 = c(rep(100,9),rep(500,9),rep(1000,9)), 
  final_database)

v1 <- ggplot(data = final_database_vf) +
  aes(x = "", y = V1) +
  geom_boxplot(shape = "circle", fill = "#EF562D") +
  facet_grid(vars(), vars(B1)) +
  scale_color_brewer(palette = "YlOrRd") + 
  labs(x="Tama�o de Muestra", y="Sesgo Relativo",title = "Sesgo Relativo para \U003B2_0") +
  geom_hline(yintercept=0.0,linetype="dashed",color="black") +
  scale_y_continuous(limits=c(-0.025,0.025)) +
  theme_minimal() +
  theme(legend.position="bottom")

v2 <- ggplot(data = final_database_vf) +
  aes(x = "", y = V1) +
  geom_boxplot(shape = "circle", fill = "#EF562D") +
  facet_grid(vars(), vars(B1)) +
  scale_color_brewer(palette = "YlOrRd") + 
  labs(x="Tama�o de Muestra", y="Sesgo Relativo",title = "Sesgo Relativo para \U003B2_1") +
  geom_hline(yintercept=0.0,linetype="dashed",color="black") +
  scale_y_continuous(limits=c(-0.025,0.025)) +
  theme_minimal() +
  theme(legend.position="bottom")

v3 <- ggplot(data = final_database_vf) +
  aes(x = "", y = V1) +
  geom_boxplot(shape = "circle", fill = "#EF562D") +
  facet_grid(vars(), vars(B1)) +
  scale_color_brewer(palette = "YlOrRd") + 
  labs(x="Tama�o de Muestra", y="Sesgo Relativo",title = "Sesgo Relativo para \U003B2_2") +
  geom_hline(yintercept=0.0,linetype="dashed",color="black") +
  scale_y_continuous(limits=c(-0.025,0.025)) +
  theme_minimal() +
  theme(legend.position="bottom")

v4 <- ggplot(data = final_database_vf) +
  aes(x = "", y = V1) +
  geom_boxplot(shape = "circle", fill = "#EF562D") +
  facet_grid(vars(), vars(B1)) +
  scale_color_brewer(palette = "YlOrRd") + 
  labs(x="Tama�o de Muestra", y="Sesgo Relativo",title = "Sesgo Relativo para \U003B2_3") +
  geom_hline(yintercept=0.0,linetype="dashed",color="black") +
  scale_y_continuous(limits=c(-0.025,0.025)) +
  theme_minimal() +
  theme(legend.position="bottom")

v5 <- ggplot(data = final_database_vf) +
  aes(x = "", y = V1) +
  geom_boxplot(shape = "circle", fill = "#EF562D") +
  facet_grid(vars(), vars(B1)) +
  scale_color_brewer(palette = "YlOrRd") + 
  labs(x="Tama�o de Muestra", y="Sesgo Relativo",title = "Sesgo Relativo para \U003B1") +
  geom_hline(yintercept=0.0,linetype="dashed",color="black") +
  theme_minimal() +
  theme(legend.position="bottom")

ggpubr::ggarrange(v1,v2,v3,v4,v5)

#### Datos reales ####

real_data_enf = lancet(3)
real_data_med = real_data_enf

library(skimr)

enf <- real_data_med %>% mutate(factor = case_when(lf == 999 ~ 1, lf == 2000 ~ 2, lf == 3000 ~ 3, lf == 4000 ~ 4,lf == 5000 ~ 5, lf == Inf ~ 6))
enf$factor <- factor(x = enf$factor,labels = c("[850-999]","[1000-2000]","[2001-3000]","[3001-4000]","[4001-5000]","[5001-Inf.]"))

ib <- enf %>% group_by(factor) %>% skim()

enf %>% group_by(factor,INSTITUCION) %>% summarise( n())



tau_seq_sim = seq(0.1,0.90,0.1)
#m1 = gamlss(Surv(li,lf,type="interval2")~.,family = WEI3ic,data = real_data_enf)
m2 = gamlss(Surv(li,lf,type="interval2")~.,family = WEI3ic,data = real_data_med)
# init_real_enf = as.vector(c(coef(m1),m1$sigma.coefficients))
init_real_med = as.vector(c(coef(m2),exp(m2$sigma.coefficients)))
abc <- confint(m2)
#### Regresi�n cuant�lica para m�dicxs ####

var_med = list()
for (j in 1:length(tau_seq_sim)) {
  var_med = append(var_med,list(reg_Wr(data = real_data_med,li = 8, lf = 9,tau_seq_sim[j],param = init_real_med)))
}

col_df <- c(colnames(model.matrix( ~ . ,subset.data.frame(real_data_med,select = -c(8,9)))),"\U03B1")
val_med <- matrix(ncol=8,nrow = 0)

for (l in 1:length(tau_seq_sim)) {
  pba <- var_med[[l]]
  val_med <- rbind(val_med,
                   cbind(pba$solution,
                         col_df,
                         tau_seq_sim[l],
                         init_real_med,
                         pba$solution + qnorm(0.975) * sqrt(diag(solve(pba$hessian))),
                         pba$solution - qnorm(0.975) * sqrt(diag(solve(pba$hessian))),
                         c(abc[,1],exp(m2$sigma.coefficients)),
                         c(abc[,2],exp(m2$sigma.coefficients))
                         ))
}
val_med <- as.data.frame(val_med)
val_med$V1 <- exp(as.numeric(val_med$V1))
val_med$V3 <- as.numeric(val_med$V3)
val_med$V5 <- exp(as.numeric(val_med$V5))
val_med$V6 <- exp(as.numeric(val_med$V6))
val_med$V7 <- exp(as.numeric(val_med$V7))
val_med$V8 <- exp(as.numeric(val_med$V8))
val_med$init_real_med <- as.numeric(val_med$init_real_med)


ggplot(val_med) +
  aes(x = V3, y = V1) +
  geom_ribbon(mapping = aes(ymin = V6,ymax = V5),fill = "#B2B9FF") +
  geom_line(size = 0.5, colour = "#1B00FF") +
  geom_ribbon(mapping = aes(ymin = V7,ymax = V8),fill = "#cccccc",alpha=0.5) +
  geom_line(mapping = aes(y=exp(init_real_med)),size=0.5) + 
  ggthemes::theme_pander() +
  facet_wrap(vars(col_df), scales = "free")

\end{lstlisting}

\section{Estimaci�n de los coeficientes - ENSUSALUD 2015}

% Please add the following required packages to your document preamble:
% \usepackage{graphicx}
\begin{table}[!ht]
\centering
\resizebox{\textwidth}{!}{%
\begin{tabular}{l|lllll}
 & \textbf{Cuantil $t$} & \textbf{Estimado} & \textbf{Error Std.} & \textbf{valor-t} & \textbf{p-valor} \\ \hline
(Intercept)             & 0.1 & 8.340  & 0.050 & 168.082 & 0.000 \\
INSTITUCIONESSALUD      & 0.1 & 0.148  & 0.011 & 13.151  & 0.000 \\
INSTITUCIONFF.AA. Y PNP & 0.1 & -0.025 & 0.038 & -0.661  & 0.508 \\
INSTITUCIONCL�NICAS     & 0.1 & -0.177 & 0.023 & -7.708  & 0.000 \\
C2P4Mujer               & 0.1 & -0.769 & 0.031 & -24.721 & 0.000 \\
C2P13No                 & 0.1 & -0.157 & 0.012 & -13.507 & 0.000 \\
C2P21                   & 0.1 & 0.007  & 0.001 & 12.826  & 0.000 \\
C2P24No                 & 0.1 & -0.314 & 0.020 & -15.712 & 0.000 \\
C2P26No                 & 0.1 & -0.188 & 0.018 & -10.586 & 0.000 \\
C2P27                   & 0.1 & 0.003  & 0.001 & 5.948   & 0.000 \\
\textbackslash{}alpha   & 0.1 & 3.213  & 0.067 & 48.181  & 0.000 \\ \hline
\end{tabular}%
}
\caption{Estimaci�n de los efectos para el cuantil $t = 0.1$.}
\label{tab:t01}
\end{table}

% Please add the following required packages to your document preamble:
% \usepackage{graphicx}
\begin{table}[]
\centering
\resizebox{\textwidth}{!}{%
\begin{tabular}{l|lllll}
 & \textbf{Cuantil $t$} & \textbf{Estimado} & \textbf{Error Std.} & \textbf{valor-t} & \textbf{p-valor} \\ \hline
(Intercept)             & 0.2 & 8.340  & 0.045 & 185.538 & 0.000 \\
INSTITUCIONESSALUD      & 0.2 & 0.163  & 0.011 & 14.781  & 0.000 \\
INSTITUCIONFF.AA. Y PNP & 0.2 & 0.029  & 0.040 & 0.717   & 0.473 \\
INSTITUCIONCL�NICAS     & 0.2 & -0.187 & 0.023 & -8.284  & 0.000 \\
C2P4Mujer               & 0.2 & -0.499 & 0.016 & -32.152 & 0.000 \\
C2P13No                 & 0.2 & -0.103 & 0.011 & -9.282  & 0.000 \\
C2P21                   & 0.2 & 0.010  & 0.001 & 16.838  & 0.000 \\
C2P24No                 & 0.2 & -0.444 & 0.023 & -19.685 & 0.000 \\
C2P26No                 & 0.2 & -0.195 & 0.017 & -11.212 & 0.000 \\
C2P27                   & 0.2 & 0.005  & 0.001 & 8.775   & 0.000 \\
\textbackslash{}alpha   & 0.2 & 3.271  & 0.048 & 67.755  & 0.000 \\ \cline{1-5}
\end{tabular}%
}
\caption{Estimaci�n de los coeficientes para el cuantil $t = 0.2$.}
\label{tab:t02}
\end{table}

% Please add the following required packages to your document preamble:
% \usepackage{graphicx}
\begin{table}[]
\centering
\resizebox{\textwidth}{!}{%
\begin{tabular}{l|lllll}
 & \textbf{Cuantil $t$} & \textbf{Estimado} & \textbf{Error Std.} & \textbf{valor-t} & \textbf{p-valor} \\ \hline
(Intercept)             & 0.3 & 8.340  & 0.041 & 204.890 & 0.000 \\
INSTITUCIONESSALUD      & 0.3 & 0.170  & 0.011 & 16.030  & 0.000 \\
INSTITUCIONFF.AA. Y PNP & 0.3 & 0.003  & 0.037 & 0.079   & 0.937 \\
INSTITUCIONCL�NICAS     & 0.3 & -0.145 & 0.022 & -6.669  & 0.000 \\
C2P4Mujer               & 0.3 & -0.432 & 0.013 & -32.109 & 0.000 \\
C2P13No                 & 0.3 & -0.112 & 0.011 & -10.522 & 0.000 \\
C2P21                   & 0.3 & 0.009  & 0.001 & 15.709  & 0.000 \\
C2P24No                 & 0.3 & -0.318 & 0.018 & -17.476 & 0.000 \\
C2P26No                 & 0.3 & -0.181 & 0.016 & -11.121 & 0.000 \\
C2P27                   & 0.3 & 0.005  & 0.001 & 9.276   & 0.000 \\
\textbackslash{}alpha   & 0.3 & 3.420  & 0.047 & 72.947  & 0.000 \\ \cline{1-5}
\end{tabular}%
}
\caption{Estimaci�n de los coeficientes para el cuantil $t = 0.3$.}
\label{tab:t03}
\end{table}

% Please add the following required packages to your document preamble:
% \usepackage{graphicx}
\begin{table}[]
\centering
\resizebox{\textwidth}{!}{%
\begin{tabular}{l|lllll}
 & \textbf{Cuantil $t$} & \textbf{Estimado} & \textbf{Error Std.} & \textbf{valor-t} & \textbf{p-valor} \\ \hline
(Intercept)             & 0.4 & 8.335  & 0.039 & 213.730 & 0.000 \\
INSTITUCIONESSALUD      & 0.4 & 0.190  & 0.011 & 18.022  & 0.000 \\
INSTITUCIONFF.AA. Y PNP & 0.4 & 0.000  & 0.037 & 0.012   & 0.991 \\
INSTITUCIONCL�NICAS     & 0.4 & -0.137 & 0.021 & -6.476  & 0.000 \\
C2P4Mujer               & 0.4 & -0.379 & 0.012 & -30.881 & 0.000 \\
C2P13No                 & 0.4 & -0.077 & 0.011 & -7.364  & 0.000 \\
C2P21                   & 0.4 & 0.009  & 0.001 & 17.174  & 0.000 \\
C2P24No                 & 0.4 & -0.310 & 0.017 & -17.869 & 0.000 \\
C2P26No                 & 0.4 & -0.178 & 0.016 & -11.338 & 0.000 \\
C2P27                   & 0.4 & 0.005  & 0.001 & 10.049  & 0.000 \\
\textbackslash{}alpha   & 0.4 & 3.418  & 0.047 & 73.225  & 0.000 \\ \hline
\end{tabular}%
}
\caption{Estimaci�n de los coeficientes para el cuantil $t = 0.4$.}
\label{tab:t04}
\end{table}

% Please add the following required packages to your document preamble:
% \usepackage{graphicx}
\begin{table}[]
\centering
\resizebox{\textwidth}{!}{%
\begin{tabular}{l|lllll}
 & \textbf{Cuantil $t$} & \textbf{Estimado} & \textbf{Error Std.} & \textbf{valor-t} & \textbf{p-valor} \\ \hline
(Intercept)             & 0.5 & 8.340  & 0.040 & 208.883 & 0.000 \\
INSTITUCIONESSALUD      & 0.5 & 0.201  & 0.011 & 18.900  & 0.000 \\
INSTITUCIONFF.AA. Y PNP & 0.5 & 0.015  & 0.037 & 0.395   & 0.693 \\
INSTITUCIONCL�NICAS     & 0.5 & -0.133 & 0.021 & -6.225  & 0.000 \\
C2P4Mujer               & 0.5 & -0.377 & 0.012 & -30.502 & 0.000 \\
C2P13No                 & 0.5 & -0.083 & 0.011 & -7.884  & 0.000 \\
C2P21                   & 0.5 & 0.009  & 0.001 & 16.511  & 0.000 \\
C2P24No                 & 0.5 & -0.301 & 0.018 & -17.028 & 0.000 \\
C2P26No                 & 0.5 & -0.173 & 0.016 & -10.852 & 0.000 \\
C2P27                   & 0.5 & 0.007  & 0.001 & 12.809  & 0.000 \\
\textbackslash{}alpha   & 0.5 & 3.399  & 0.046 & 73.212  & 0.000 \\ \hline
\end{tabular}%
}
\caption{Estimaci�n de los coeficientes para el cuantil $t = 0.5$.}
\label{tab:t05}
\end{table}

% Please add the following required packages to your document preamble:
% \usepackage{graphicx}
\begin{table}[]
\centering
\resizebox{\textwidth}{!}{%
\begin{tabular}{l|lllll}
 & \textbf{Cuantil $t$} & \textbf{Estimado} & \textbf{Error Std.} & \textbf{valor-t} & \textbf{p-valor} \\ \hline
(Intercept)             & 0.6 & 8.340  & 0.040 & 211.331 & 0.000 \\
INSTITUCIONESSALUD      & 0.6 & 0.206  & 0.011 & 19.505  & 0.000 \\
INSTITUCIONFF.AA. Y PNP & 0.6 & 0.043  & 0.038 & 1.120   & 0.263 \\
INSTITUCIONCL�NICAS     & 0.6 & -0.110 & 0.022 & -5.087  & 0.000 \\
C2P4Mujer               & 0.6 & -0.339 & 0.012 & -28.793 & 0.000 \\
C2P13No                 & 0.6 & -0.063 & 0.011 & -6.000  & 0.000 \\
C2P21                   & 0.6 & 0.009  & 0.001 & 16.770  & 0.000 \\
C2P24No                 & 0.6 & -0.297 & 0.018 & -17.014 & 0.000 \\
C2P26No                 & 0.6 & -0.171 & 0.016 & -10.904 & 0.000 \\
C2P27                   & 0.6 & 0.008  & 0.001 & 14.247  & 0.000 \\
\textbackslash{}alpha   & 0.6 & 3.421  & 0.047 & 73.494  & 0.000 \\ \hline
\end{tabular}%
}
\caption{Estimaci�n de los coeficientes para el cuantil $t = 0.6$.}
\label{tab:t06}
\end{table}

% Please add the following required packages to your document preamble:
% \usepackage{graphicx}
\begin{table}[]
\centering
\resizebox{\textwidth}{!}{%
\begin{tabular}{l|lllll}
 & \textbf{Cuantil $t$} & \textbf{Estimado} & \textbf{Error Std.} & \textbf{valor-t} & \textbf{p-valor} \\ \hline
(Intercept)             & 0.7 & 8.340  & 0.041 & 203.767 & 0.000 \\
INSTITUCIONESSALUD      & 0.7 & 0.218  & 0.011 & 19.819  & 0.000 \\
INSTITUCIONFF.AA. Y PNP & 0.7 & 0.065  & 0.041 & 1.592   & 0.111 \\
INSTITUCIONCL�NICAS     & 0.7 & -0.077 & 0.024 & -3.261  & 0.001 \\
C2P4Mujer               & 0.7 & -0.333 & 0.012 & -27.463 & 0.000 \\
C2P13No                 & 0.7 & -0.049 & 0.011 & -4.440  & 0.000 \\
C2P21                   & 0.7 & 0.010  & 0.001 & 17.598  & 0.000 \\
C2P24No                 & 0.7 & -0.303 & 0.018 & -16.681 & 0.000 \\
C2P26No                 & 0.7 & -0.147 & 0.016 & -9.271  & 0.000 \\
C2P27                   & 0.7 & 0.008  & 0.001 & 14.765  & 0.000 \\
\textbackslash{}alpha   & 0.7 & 3.271  & 0.045 & 72.982  & 0.000 \\ \hline
\end{tabular}%
}
\caption{Estimaci�n de los coeficientes para el cuantil $t = 0.7$.}
\label{tab:t07}
\end{table}

% Please add the following required packages to your document preamble:
% \usepackage{graphicx}
\begin{table}[]
\centering
\resizebox{\textwidth}{!}{%
\begin{tabular}{l|lllll}
 & \textbf{Cuantil $t$} & \textbf{Estimado} & \textbf{Error Std.} & \textbf{valor-t} & \textbf{p-valor} \\ \hline
(Intercept)             & 0.8 & 8.340  & 0.039 & 211.625 & 0.000 \\
INSTITUCIONESSALUD      & 0.8 & 0.248  & 0.011 & 22.799  & 0.000 \\
INSTITUCIONFF.AA. Y PNP & 0.8 & 0.032  & 0.037 & 0.870   & 0.384 \\
INSTITUCIONCL�NICAS     & 0.8 & -0.095 & 0.022 & -4.426  & 0.000 \\
C2P4Mujer               & 0.8 & -0.321 & 0.012 & -27.711 & 0.000 \\
C2P13No                 & 0.8 & -0.043 & 0.011 & -4.006  & 0.000 \\
C2P21                   & 0.8 & 0.010  & 0.001 & 17.883  & 0.000 \\
C2P24No                 & 0.8 & -0.277 & 0.017 & -16.122 & 0.000 \\
C2P26No                 & 0.8 & -0.160 & 0.016 & -10.327 & 0.000 \\
C2P27                   & 0.8 & 0.010  & 0.001 & 17.205  & 0.000 \\
\textbackslash{}alpha   & 0.8 & 3.388  & 0.047 & 72.508  & 0.000 \\ \hline
\end{tabular}%
}
\caption{Estimaci�n de los coeficientes para el cuantil $t = 0.8$.}
\label{tab:t08}
\end{table}

% Please add the following required packages to your document preamble:
% \usepackage{graphicx}
\begin{table}[]
\centering
\resizebox{\textwidth}{!}{%
\begin{tabular}{l|lllll}
 & \textbf{Cuantil $t$} & \textbf{Estimado} & \textbf{Error Std.} & \textbf{valor-t} & \textbf{p-valor} \\ \hline
(Intercept)             & 0.9 & 8.898  & 0.046 & 192.163 & 0.000 \\
INSTITUCIONESSALUD      & 0.9 & 0.167  & 0.011 & 14.781  & 0.000 \\
INSTITUCIONFF.AA. Y PNP & 0.9 & -0.019 & 0.039 & -0.480  & 0.631 \\
INSTITUCIONCL�NICAS     & 0.9 & -0.179 & 0.023 & -7.908  & 0.000 \\
C2P4Mujer               & 0.9 & -0.420 & 0.014 & -29.886 & 0.000 \\
C2P13No                 & 0.9 & -0.103 & 0.011 & -9.122  & 0.000 \\
C2P21                   & 0.9 & 0.009  & 0.001 & 15.493  & 0.000 \\
C2P24No                 & 0.9 & -0.332 & 0.020 & -16.599 & 0.000 \\
C2P26No                 & 0.9 & -0.258 & 0.019 & -13.315 & 0.000 \\
C2P27                   & 0.9 & 0.007  & 0.001 & 11.407  & 0.000 \\
\textbackslash{}alpha   & 0.9 & 3.289  & 0.050 & 66.477  & 0.000 \\ \hline
\end{tabular}%
}
\caption{Estimaci�n de los coeficientes para el cuantil $t = 0.9$.}
\label{tab:t09}
\end{table}


% ---------------------------------------------------------------------------- %
% Bibliografia
\backmatter \singlespacing   % espacio simple

\renewcommand{\harvardand}{y} % cambiar "and" por "y" al generar la bibliografia.
\bibliography{bibliografia}
\bibliographystyle{dcu}

\pagestyle{blank}
\end{document}

