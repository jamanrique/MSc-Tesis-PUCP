%% ------------------------------------------------------------------------- %%
\chapter{Introducci�n}
\label{cap:introduccion}

%% ------------------------------------------------------------------------- %%
\section{Consideraciones Preliminares}
\label{sec:consideraciones}

En muchas situaciones pr�cticas deseamos investigar como ciertas variables influyen en una variable continua que asume valores en el intervalo $(0,1)$, tales como, porcentajes, proporciones, tasas, etc. Por ejemplo, la tasa de desnutrici�n de una cierta regi�n puede ser influenciada por el PBI, o la fracci�n del gasto de un hogar destinado a alimentos puede ser influenciada por variables como el tama�o de la familia, el ingreso total de la familia, etc. En estos casos, los modelos de regresi�n pueden no ser apropiados para modelar este tipo de datos porque la variable respuesta s�lo toma valores en un rango limitado y los valores estimados pueden caer fuera del rango. Una posible soluci�n es transformar este tipo de datos para que asuman valores en toda la recta y modelarlos mediante una regresi�n. Este enfoque puede presentar varios problemas, uno de ellos es que los par�metros del modelo no sean f�cilmente interpretados en t�rminos de los datos originales. Otro problema es que generalmente las proporciones presentan asimetr�a, por lo tanto la suposici�n de normalidad no seria adecuada.\footnote{Pie de pagina de ejemplo.}


%% ------------------------------------------------------------------------- %%
\section{Objetivos}
\label{sec:objetivo}


El objetivo general de la tesis es estudiar propiedades, estimar y aplicar a conjuntos de datos reales el
modelo de regresi�n beta desde el punto de vista de la estad�stica bayesiana. De manera espec�fica:
\begin{itemize}
\item Revisar la literatura acerca de los diferentes propuestas de modelos de regresi�n para proporciones.
\item Proponer estudiar, propiedades, e implementar la estimaci�n del modelo de regresi�n beta desde la perspectiva bayesiana.
\item Realizar estudios de simulaci�n de computaci�n intensiva aprovechando el uso de un GRID computacional (proyecto LEGION).
\item Aplicar el modelo a conjunto de datos reales.
\end{itemize}

%% ------------------------------------------------------------------------- %%
\section{Organizaci�n del Trabajo}
\label{sec:organizacion}

En el Cap�tulo \ref{cap:conceptos}, presentamos conceptos previos al desarrollo de ...

Finalmente, en el Cap�tulo \ref{cap:conclusiones} discutimos algunas conclusiones obtenidas en este trabajo. Analizamos la ventajas y desventajas de los m�todos propuestos,...

En el anexo presentamos algunas pruebas de resultados en m�s detalles (Ap�ndice \ref{ape:demostraciones}) y tambi�n los programas utilizadas en las aplicaciones a conjuntos de datos reales.


