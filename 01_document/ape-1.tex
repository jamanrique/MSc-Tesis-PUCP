\chapter{Ap�ndice}

\section{Pseudoc�digo de la simulaci�n}
\label{seudo}

\begin{lstlisting}
Simulamos valores de las siguientes distribuciones:

Definimos los siguientes valores:
N = [100, 500, 1000]
B = [7, 0.3, 0.84, 2.5] 
Sigma = 2
t=[0.1, 0.2, 0.3, 0.4, 0.5, 0.6, 0.7, 0.8, 0.9]
M = 5000

Para cada cuantil en t:
	Para cada n en N:
		Para cada replica en M:
		1 Simular n valores de las siguientes distribuciones:
			X1 ~ Beta(2,3) 
			X2 ~ Normal(2,0.5)
			X3 ~ Gamma(2,25)
		2 Generar la funci�n de enlace:
			Qt = exp(B[1] + B[2]*X1 + B[3]*X2 + B[4]*X3)
		3 Para cada i en n:
			Simular 1 valor de la siguiente distribucion:
			Y[i] ~ W_r(Qt[i], Sigma, cuantil)
		4 Censurar la variable Y de forma intervalar tal que
			Z ~ Categorica
		5 Obtener los limites inferiores y superiores de
			cada categoria de Z
		6 Crear la base de datos simulada
			df <- [L_inf, L_sup, X1, X2, X3]
		7 Ejecutar la regresion de censura intervalar
		8 Guardar los resultados
\end{lstlisting}
