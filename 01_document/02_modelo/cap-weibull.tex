\chapter{Distribución Weibull}

El presente capítulo tiene el objetivo de proponer una reparametrización de la distribución Weibull para adaptarla al modelo de regresión cuantílica. Se definirán su función de densidad y función acumulada, y asimismo se estudiarán sus propiedades.

\section{Distribución Weibull}

\subsection{Función de densidad}

Una variable aleatoria continua $Y$, con soporte $Y \in [0,\infty]$, sigue una distribución Weibull si su función de densidad, la cual tiene parámetro de forma $\alpha$ y dispersión $\sigma$, es dada por la siguiente expresión:

\begin{equation*}
f{(y)}= \frac{\alpha}{\sigma}\left(\frac{y}{\sigma}\right)^{\alpha-1} \exp{\left(-\frac{y}{\sigma}\right)^{\alpha}}
\end{equation*}
\noindent en dónde $\alpha > 0$ y $\sigma > 0$. La notación de una variable aleatoria $Y$ que sigue esta distribución se indica como $Y \sim W(\alpha,\sigma)$. Asimismo, la función acumulada de $Y$ corresponde a la siguiente expresión:

\[F{(y)}=1-\exp^{-{(\frac{y}{\sigma})}^{\alpha}}.\]

y su función de cuantiles es dada por:

\[q_{t}=\sigma{(-\log{(1-t)})}^{\frac{1}{\alpha}}\]

\noindent para $0 < t < 1$.

Para dicha variable aleatoria $Y$ la media y varianza es de la siguiente forma:

\[E(Y)=\sigma \Gamma\left( 1+\frac{1}{\alpha} \right).\]
\[V(Y)=\sigma^{2}\left[ \Gamma\left( 1+\frac{2}{\alpha} \right)-\left( \Gamma\left( 1+\frac{1}{\alpha} \right) \right)^{2} \right].\]

\subsection{Proposición de una nueva estructura de la distribución}

Consideramos, para la distribución Weibull, una reparametrización en términos del cuantil $t, q_{t}$, dada por:

\[q_{t}=\sigma\left( -\log\left( 1-t \right) \right)^{\frac{1}{\alpha}}.\]

Al respecto, cabe indicar que $t$ será un valor conocido y se encuentra en el intervalo $[0,1]$. En esta nueva esctructura, $q_{t}$ y $\alpha$ tienen espacios paramétricos independientes tal que $(q_{t},\alpha) \in (0,\infty)$ x $(0,\infty)$. Una variable aleatoria que sigue esta parametrización se denota como $Y \sim W_{r}(q_{t},\alpha)$.

La función de densidad de dicha variable $Y$ tiene la siguiente expresión:
\begin{equation}
f_{Y}(y| q_{t},\alpha)=\frac{\alpha c(t)}{q_{t}}\left( \frac{y}{q_{t}} \right)^{\alpha-1}\exp\left( -c(t)\left( \frac{y}{q_{t}} \right)^{\alpha} \right)
\end{equation}
\\
\noindent en dónde $c(t)= \left( -\log(1-t) \right)^{\frac{1}{\alpha}}$. Los parámetros $q_{t}$ y $\alpha$ caracterizan la función de densidad conforme se observa el gráfico siguiente:

\textbf{Nota del profesor Valdivieso: Mejorar esta densidad.}

\begin{figure}[H]
\includegraphics[width=\textwidth]{densidad1}
\caption{Función de densidad de una distribución Weibull bajo la reparametrización propuesta.}
\end{figure}

\noindent Se observa que en la medida que $q_{t}$ aumenta, la distribución incrementa su asimetría hacia la derecha. Ello también sucede, aunque en menor grado, cuando $\alpha$ aumenta. No obstante, se observa que en la medida que $\alpha$ tiende a 0, incrementa la dispersión.

Reexpresando la función acumulada en los términos de la parametrización propuesta, esta tendría la siguiente forma:
\begin{equation} \label{eq:1}
F_{Y}\left(y| q_{t},\alpha,t \right)=1-\exp\left( -c(t)\left( \frac{y}{q_{t}} \right)^{\alpha} \right).
\end{equation}
\subsection{Estudio de la parametrización propuesta}

La esperanza y varianza de una variable aleatoria bajo la parametrización Weibull propuesta están dadas bajo la siguiente expresión:
\begin{equation}
E(Y)=\frac{q_{t}}{c(t)^{\frac{1}{\alpha}}}\Gamma\left( 1+\frac{1}{\alpha} \right)
\end{equation}

\begin{equation}
Var(Y)=\frac{q_{t}^{2}}{c(t)^{\frac{1}{\alpha}}}\left[ \Gamma\left( 1+\frac{2}{\alpha}\right)-\Gamma\left( 1+\frac{1}{\alpha} \right)^{2} \right]
\end{equation}

Bajo la parametrización propuesta, se observa que para un $\alpha$ fijo el valor esperado se comporta de forma lineal en la medida que aumente el parámetro $q_{t}$ conforme se observa en el cuadro siguiente:

\textbf{Nota del profesor Valdivieso: Mejorar esta densidad.}

\begin{figure}[H]
	\includegraphics[width=\textwidth]{esperado}
	\caption{Valor esperado de una distribución Weibull bajo la parametrización propuesta.}
\end{figure}
\noindent No obstante, para un $q_{t}$ fijo, lo mismo no se observa en la medida que aumente $\alpha$. Se observa un comportamiento no lineal y asintótico: cuando $\alpha$ tiende a 0, el valor esperado tiende a infinito. Cuando $\alpha$ aumenta, el valor esperado se estabiliza.

En el caso de la varianza se observa que para un $\alpha$ fijo, en la medida que aumente el parámetro $q_{t}$ la varianza aumenta de forma exponencial. No obstante, y como se puede apreciar cuando $q_{t}$ está fijo, en la medida que los valores de $\alpha$ sean pequeños, la varianza incrementa drásticamente. Asimismo, como se aprecia en el cuadro adjunto, la varianza tiende a 0 en la medida que $\alpha$ aumente.

\begin{figure}[H]
	\includegraphics[width=\textwidth]{varianza}
	\caption{Varianza bajo la parametrización propuesta.}
\end{figure}
